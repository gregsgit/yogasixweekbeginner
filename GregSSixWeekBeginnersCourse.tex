\documentclass{book}

\usepackage{hyperref}
\usepackage[pdftex]{graphicx}
\usepackage{enumitem}
\usepackage{textcomp}   % for degree symbol
\usepackage{wrapfig}
\usepackage[dvipsnames]{xcolor}


% Pose shortcodes
\newcommand{\pose}[1]{\emph{#1}}
%
\newcommand{\ams}{\pose{Adho Mukha Svanasana}}
\newcommand{\ardhal}{\pose{Ardha Halasana}}
\newcommand{\ardchand}{\pose{Ardha Chandrasana}}
\newcommand{\badd}{\pose{Baddhanguliyasana}}
\newcommand{\badhasttad}{\pose{Baddha Hasta Tadasana}}
\newcommand{\badhastutt}{\pose{Baddha Hasta Uttanasana}}
\newcommand{\dand}{\pose{Dandasana}}
\newcommand{\ekapadsarv}{\pose{Eka Pada Sarvangasana}}
\newcommand{\gomu}{\pose{Gomukasana}}
\newcommand{\hal}{\pose{Halasana}}
\newcommand{\karn}{\pose{Karnapidasana}}
\newcommand{\nam}{\pose{Namaskarasana}}
\newcommand{\padadand}{\pose{Padanghusta Dandasana}}
\newcommand{\padang}{\pose{Padangusthasana}}
\newcommand{\parshastpad}{Parsva Hasta Padasana}
\newcommand{\parsvo}{\pose{Parsvottanasana}}
\newcommand{\paschi}{\pose{Paschimottanasana}}
\newcommand{\paschbadhast}{\pose{Paschima Baddha Hastasana}}
\newcommand{\paschnama}{\pose{Paschima Namaskarasana}}
\newcommand{\praspad}{\pose{Prasarita Padottanasana}}
\newcommand{\sarv}{\pose{Salamba Sarvangasana}}
\newcommand{\sav}{\pose{Savasana}}
\newcommand{\setubandsarv}{\pose{Setu Bandha Sarvangasana}}
\newcommand{\tad}{\pose{Tadasana}}
\newcommand{\urdbad}{\pose{Urdhva Baddhanguliyasana}}
\newcommand{\urdhast}{\pose{Urdhva Hastasana}}
\newcommand{\urdhastdand}{\pose{Urdhva Hasta Dandasana}}
\newcommand{\urdnam}{\pose{Urdhva Namaskarasana}}
\newcommand{\utka}{\pose{Utkatasana}}
\newcommand{\utt}{\pose{Uttanasana}}
\newcommand{\utthastpad}{\pose{Utthita Hasta Padasana}}
\newcommand{\uttparsva}{\pose{Utthita Parsvakonasana}}
\newcommand{\utttrik}{\pose{Utthita Trikonasana}}
\newcommand{\vim}{\pose{Vimanasana}}
\newcommand{\virai}{\pose{Virabhdrasana I}}
\newcommand{\viraii}{\pose{Virabhadrasana II}}
\newcommand{\vrk}{\pose{Vrksasana}}

\newcommand{\poseFig}[1]{
  \begin{wrapfigure}{r}{2.1in}
      \begin{center}
        \includegraphics[width=2.0in]{Figures/{"#1"}.jpg}
      \end{center}
    \end{wrapfigure}
}

\newcommand{\amsFig}{\poseFig{Adho Mukha Svanasana}}
\newcommand{\ardhalFig}{\poseFig{Ardha Halasana}}
\newcommand{\ardchandFig}{\poseFig{Ardha Chandrasana}}
\newcommand{\baddFig}{\poseFig{Baddhanguliyasana}}
\newcommand{\badhasttadFig}{\poseFig{Baddha Hasta Tadasana}}
\newcommand{\badhastuttFig}{\poseFig{Baddha Hasta Uttanasana}}
\newcommand{\dandFig}{\poseFig{Dandasana}}
\newcommand{\ekapadsarvFig}{\poseFig{Eka Pada Sarvangasana}}
\newcommand{\gomuFig}{\poseFig{Gomukasana}}
\newcommand{\halFig}{\poseFig{Halasana}}
\newcommand{\karnFig}{\poseFig{Karnapidasana}}
\newcommand{\namFig}{\poseFig{Namaskarasana}}
\newcommand{\padadandFig}{\poseFig{Padanghusta Dandasana}}
\newcommand{\padangFig}{\poseFig{Padangusthasana}}
\newcommand{\parshastpadFig}{Parsva Hasta Padasana}
\newcommand{\parsvoFig}{\poseFig{Parsvottanasana}}
\newcommand{\paschiFig}{\poseFig{Paschimottanasana}}
\newcommand{\paschbadhastFig}{\poseFig{Paschima Baddha Hastasana}}
\newcommand{\paschnamaFig}{\poseFig{Paschima Namaskarasana}}
\newcommand{\praspadFig}{\poseFig{Prasarita Padottanasana}}
\newcommand{\sarvFig}{\poseFig{Salamba Sarvangasana}}
\newcommand{\savFig}{\poseFig{Savasana}}
\newcommand{\setubandsarvFig}{\poseFig{Setu Bandha Sarvangasana}}
\newcommand{\tadFig}{\poseFig{Tadasana}}
\newcommand{\urdbadFig}{\poseFig{Urdhva Baddhanguliyasana}}
\newcommand{\urdhastFig}{\poseFig{Urdhva Hastasana}}
\newcommand{\urdhastdandFig}{\poseFig{Urdhva Hasta Dandasana}}
\newcommand{\urdnamFig}{\poseFig{Urdhva Namaskarasana}}
\newcommand{\utkaFig}{\poseFig{Utkatasana}}
\newcommand{\uttFig}{\poseFig{Uttanasana}}
\newcommand{\utthastpadFig}{\poseFig{Utthita Hasta Padasana}}
\newcommand{\uttparsvaFig}{\poseFig{Utthita Parsvakonasana}}
\newcommand{\utttrikFig}{\poseFig{Utthita Trikonasana}}
\newcommand{\vimFig}{\poseFig{Vimanasana}}
\newcommand{\viraiFig}{\poseFig{Virabhdrasana I}}
\newcommand{\viraiiFig}{\poseFig{Virabhadrasana II}}
\newcommand{\vrkFig}{\poseFig{Vrksasana}}



% \PC{week}{pose #}
\newcommand{\PC}[2]{\hfill(Week #1, Pose #2 in PC)}

% \term{xxx} - introduce a standard Iyengar word or phrase
\newcommand{\term}[1]{``{#1}''}

% \newpose{xxx} - first time this pose has been used in course
\newcommand{\newpose}[1]{{\color{Blue}{#1}}}

\newcounter{week}

% list of poses for a week
\newlist{poselist}{enumerate}{1}
\setlist[poselist]{label=w\arabic{week}.\arabic*}

\setlength{\parindent}{0cm}
\setlength{\parskip}{1ex}


\title{Six Week Iyengar Yoga Course for Beginners}
\author{Greg Sullivan \href{mailto:gregs@sulliwood.org}{gregs@sulliwood.org} \\
For Heart of Iyengar Yoga Teacher Training course with Peentz Dubble
}
\date{March 11, 2017}

\begin{document}

\maketitle

\tableofcontents

\chapter*{Overview / Goals}
\label{chap:overview}

\addcontentsline{toc}{chapter}{Overview/Goals}

The overriding goal of a class for beginning students is to instill an
enjoyment and appreciation of yoga practice. The goal is to give
beginning students a taste for the benefits of yoga practice, and to,
as much as possible, give these beginning students tools that they can
use to safely and joyfully continue to practice yoga. As Geeta Iyengar
writes in the preface of ``Yoga in Action: A Preliminary
Course''\cite{GeetaPC}, 

\begin{quote}
  \emph{Yoga in Action for Beginners is not the end, but the beginning of
  yoga. It is for the practitioner to ignite the hidden force of yoga
  from within, so that it throws the Light on the path of the yogic
  journey.}
\end{quote}

Of course, Preliminary Course gives a 28-week syllabus, and for this
assignment we are creating only a 6-week course.

The immediate goals of the first few classes is to help students
familiarize themselves with their bodies, and also with the
terminology we use to describe motions and poses in Iyengar Yoga
classes.

Yoga, and in particular asana practice, is not about how you look. If
your pose is not geometrically perfect, you are not a failure. One of
the lessons that yoga teaches is vairagya, detachment. Along with
detachment is non-judgement. Judging yourself, in an asana, as ``good''
or ``bad'' is not helpful. The question is ``what is the goal?'' How do
you strive towards the goal? In this way, asana practice is a
microcosm of ``life practice''. The concept of ``bringing yoga off the
mat'' is important


I have mostly followed the sequences given for the first six weeks in
Geeta Iyengar's \emph{Preliminary Course}\cite{GeetaPC}, abbreviated
``PC''. For each pose given, I indicate whether it was in that week's
sequence in PC, and, if so, which number in the sequence in PC.
For example, if the fourth pose in week 2 is \nam{}, and it is also
the fourth pose in the week 2 syllabus in PC, I will write:

w2.4 \nam{} \PC{2}{4}

\section*{Formatting Notes}
\label{chap:formatting}

\addcontentsline{toc}{section}{Formatting Notes}

Pose names are italicized, as in \pose{this is a pose name}.

A new pose (not introduced earlier) will be both italicized and
colored blue, as in \newpose{\pose{this is a new pose}}.

Images of poses are taken from \emph{Preliminary Course}, as I do not
yet have images of myself in the poses.

\chapter{Week 1}
\label{chap:week1}

\setcounter{week}{1}

\textbf{Themes}: What is yoga? Your body in space.

\textbf{Reading}: Something that relates asana to the rest of one's
life; how looking for alignment and balance in asana can be brought
``off the mat'' to bring alignment and balance in the rest of one's
life.

\section{Sequence for Week 1}
\label{chap:seq1}

\begin{poselist}
\item \newpose{\sav{}} \hfill(not given as first pose in PC)
  \amsFig
  \begin{itemize}
  \item Have several blankets available if needed.
  \item Lie flat; legs together, toes pointing towards ceiling. Knees
    pointing to ceiling. Hips level. Feel what parts of your body are
    touching the ground. Is there equal weight on both heels? Both
    buttocks? Shoulder blades?
  \item Now bring your arms over your head. Can you reach all the way
    to the ground? Use blankets if need support. How close are hands;
    can you bring them closer together? Rotate your arms so that your
    palms face the ceiling. Now reverse the rotation; which causes the
    shoulder blades to separate? Try rolling the upper arms (nearer
    the shoulders) \term{inward} so that palms turn toward floor. Now keep
    upper arms turned in, but make palms parallel.
  \item Now try to increase the distance between top of shoulders and
    ears (i.e. lower shoulders) while keeping arms straight above
    head.
  \end{itemize}
\item \newpose{\tad{}}, against a wall \PC{1}{1}
  \begin{itemize}
  \item Now we are going to attempt to replicate the pose we just did
    on the floor, with the help of gravity, while standing.
  \item In \sav{}, the floor was a useful reference point -- we
    could tell if our hips were even, or our shoulder blades were
    even, by feeling the contact with the floor.
  \item Stand with your back lightly touching the wall; your heels as
    close to the wall as possible.
  \item Have your feet together.
  \item Is your weight balanced on your feet, left vs. right? Try
    shifting all your weight to your left foot; now the right foot.
  \item Is your weight evenly balanced forward and backward? That is,
    is the weight on the balls of your feet the same as the weight on
    your heels? Try moving forward, putting more weight on the balls
    of the feet. Now try lifting the toes, putting all the weight on
    the heels. Now back to even.
  \item Consider your left foot. Is the weight evenly balanced on the
    ``4 corners'' of the foot? Same for right foot.
  \item Remember what was touching the floor in Savasana? Lightly
    touching the wall, see if you can replicate the touchpoints from
    Savasana. Your heels, buttocks', shoulder blades, and back of
    head.
  \end{itemize}
\item \newpose{\urdhast{}} \PC{1}{2}
  \begin{itemize}
  \item Now raise your arms above your head. This is called \urdhast{}
    (arms). Recall where they were when you were on your back; can you
    position them the same way without gravity? Turn your arms
    ``inward'' again, so that palms turn towards wall. Now keep upper
    arms turned, but make palms parallel.
  \item Lower shoulders away from ears.
  \end{itemize}

\item \newpose{\urdbad{}} \PC{1}{3}
  \begin{itemize}
  \item Bring your arms down to pointing straight in front of you.
  \item Clasp your hands. Note which thumb is on top. 
  \item Separate your wrists and rotate the thumb sides down. This is
    \badd{}.
  \item Raise your arms, keeping your hands in \badd{}; Straighten
    your arms, especially the elbows.
  \item Lower your arms, change the interlock of your fingers so that
    other thumb is on top, repeat.
  \end{itemize}

\item	\newpose{\nam{}} \PC{1}{4}
\item	\newpose{\urdnam{}} from \urdhast{} \PC{1}{5}
\end{poselist}

\textbf{Interlude}. So far everything has been perfectly straight and
balanced, left to right, top to bottom. Now we're going to branch out.

\begin{poselist}[resume]
\item \newpose{Half \utt} (made up - not in PC)
  \begin{itemize}
  \item Bring mats to wall.
  \item Stand legs-distance from wall. Lean over so that torso is
    parallel to floor, forming right angle; hands touching wall. Above
    waist is Urdhva Hastasana (rotated 90\textdegree). Below waist is \tad.
  \end{itemize}
\item \newpose{\utthastpad} \PC{1}{6}
  \begin{itemize}
  \item Left foot to wall.
  \item Separate feet 4-5 feet apart. Arms out, parallel to
    floor. Even balance front-to-back, left to right.
  \item Lift trunk and chest. Lower shoulders
  \end{itemize}
\item \newpose{\parshastpad} \PC{1}{7}
  \begin{itemize}
  \item Turn right leg out 90\textdegree. Right heel in line with
    center of left foot (both feet centered on mat). Toes pointing
    90\textdegree, left knee pointing 90\textdegree. Turn left foot in
    slightly. Everything else unchanged. Hips still parallel to long
    edge of mat; torso facing forward; face facing forward. Are hips
    even (same height)?
  \end{itemize}
\item \newpose{\utttrik} \PC{1}{8}
  \begin{itemize}
  \item Extend the right arm out as you bend at the waist and bring
    the right arm down to the shin. Place your left arm on your left
    waist. Keep your torso perpendicular to the ground, requiring
    rotation right-to-left of the torso. Now bring left arm up,
    pointing to ceiling; two arms going in opposite directions. Lower
    shoulders (increase distance between ears and shoulders). Are hips
    perpendicular to the wall?
  \item Come up, turn feet parallel.
  \item Optional variation: Can try parallel to wall, so that wall
    gives feedback to buttocks, shoulder blades, and head about
    alignment.
  \end{itemize}
\end{poselist}

(repeat Utthita Hasta Padasana, Parsva Hasta Padasana, Utthita
Trikonasana) on opposite side (left foot to the wall).

\begin{poselist}[resume]
\item \newpose{\parsvo} (concave back) \PC{1}{9}
  \begin{itemize}
  \item Grab two blocks.
  \item Left foot to wall, Parsva Hasta Padasana. Now turn back foot
    in more, and rotate hips to parallel wall. Hands to hips. Balance
    pelvis - front-to-back and side-to-side. Chest facing front of
    mat, head facing straight ahead.
  \item Extend back, make concave, look up.
  \item Now bend so that torso is parallel to floor (recall
    half-uttanasana).
  \item Put blocks at whatever height allows you to keep concave back.
  \item Repeat with right foot to wall.
  \end{itemize}
\item \newpose{\praspad} (concave back) \PC{1}{10}
  \begin{itemize}
  \item Wide separation of feet. Feet parallel to each other; even
    w.r.t. distance from edge of mat.
  \item Bend at waist so that torso is parallel to floor. Back
    concave. Hands directly below shoulders on floor. Use blocks if
    needed to get torso parallel to floor.
  \item Hips over (in line with) feet. Weight even front-to-back on
    feet. Feet flat, weight even inner-to-outer on feet.
  \end{itemize}
\item \newpose{\dand}  \PC{1}{11}
\item \newpose{\urdhastdand} \PC{1}{12}
\item \newpose{\padadand} \PC{1}{13}
  \begin{itemize}
  \item Use a belt for integrity of back.
  \item Concave back
  \end{itemize}
\item \newpose{\paschi} \PC{1}{14}
  \begin{itemize}
  \item This is a good time to remind students to not be attached to
    the perfect form/geometry of the asana, but to consider the goals.
  \end{itemize}
\item \sav
\end{poselist}


\chapter{Week 2}
\label{chap:week2}

\setcounter{week}{2}

\textbf{Themes}: 

\textbf{Reading}:


\section{Sequence for Week 2}
\label{chap:seq2}

\begin{poselist}
\item \tad{} \PC{2}{1}
\item \urdhast{} \PC{2}{2}
\item \urdbad{} \PC{2}{3}
\item \nam{} \PC{2}{4}
\item \urdnam{} \PC{2}{5}
\item \utthastpad{} \PC{2}{6}
\item \parshastpad{} \PC{2}{7}
\item \utttrik{} \PC{2}{8}
\item \newpose{\viraii} \PC{2}{9}
  \begin{itemize}
  \item Start in \parshastpad{} - as in start of \utttrik{}.
  \item (do pose with hands on waist first, then with arms extended)
  \item Keeping everything stable, bend right knee to 90\textdegree
  \item Extend inner thigh; pull back outer thigh.
  \item Look over front arm.
  \item Repeat on other side
  \end{itemize}
\item \newpose{\uttparsva} \PC{2}{10}
  \begin{itemize}
  \item Again, start in \parshastpad{}
  \item Again, bend right knee to 90\textdegree
  \item Left hand on waist.
  \item Bring right hand to floor, or block, keeping chest facing wall
    in front of you (perpendicular to floor).
  \item Rotate abdomen.
  \item Raise left arm towards ceiling.
  \item Rotate left arm as in \utthastpad{}
  \item See if you can bring arm down alongside ear.
  \end{itemize}
\item \parsvo{} (standing, then concave back) \PC{2}{11}
\item \newpose{\parsvo{}} \PC{2}{12}
  \begin{itemize}
  \item Bend at waist, sternum directly over front knee, put hands on
    blocks (at highest setting), keeping back concave.
  \item Now walk blocks forward, extending back.
  \item Finally, bring head down over knee. 
  \end{itemize}
\item \praspad{} \PC{2}{13}
\item \dand{} \PC{2}{14}
\item \urdhastdand{} \PC{2}{15}
\item \padadand{} \PC{2}{16}
\item \newpose{\ardhal} to chair (P.51) \PC{2}{17}
  \begin{itemize}
  \item Get 4 blankets, a block, and a chair
  \item 3 blankets - set up as for Sarvangasana. 
  \item Chair at head, 1 blanket on chair. block for tailbone.
  \item Shoulders 2 in.s from edge of blankets.
  \item Swing up.
  \item Goals: hips over head, legs parallel to floor.
  \item Arm positions for Sarvangasna (w/o belts). Elbows down and
    in. Hands on mid-to-upper back. Use hands to lift hips and
    straighten back.
  \end{itemize}
\item \paschi{} \PC{2}{18}
\item \sav{} \PC{2}{19}
  
\end{poselist}


\chapter{Week 3}
\label{chap:week3}

\setcounter{week}{3}

\textbf{Themes}: 

\textbf{Reading}: 

\section{Sequence for Week 3}
\label{chap:seq3}

\begin{poselist}
\item \tad{} \PC{3}{1}
\item \urdbad{} \PC{3}{2}
\item \newpose{vrk{}} \PC{3}{3}
  \begin{itemize}
  \item Stand with back to wall, far enough away so that can reach
    back to touch wall for balance.
  \item Bend the right knee, keeping left leg in Tadasana, grab right
    foot with right hand.
  \item Place right foot high on inside of left thigh.
  \item Push thigh out against foot, foot in against thigh.
  \item Push knee back - lengthen inner right thigh, pull in outer
    right thigh. 
  \item Can you get your knee parallel with wall (perpendicular to
    gaze)? 
  \item Balance free of wall
  \item Can you raise your left hand straight up?
  \item Now your right hand?
  \item Can you bring your hands together into Urdhva Namaskarasana?
  \item Lower arms, lower right leg, repeat with left leg.
  \end{itemize}
\item \utttrik{} \PC{3}{4}
\item \viraii{} \PC{3}{5}
\item \uttparsva{} \PC{3}{6}
\item \newpose{\virai} (turning the trunk) \PC{3}{7}
  \begin{itemize}
  \item Stand left foot to wall. Turn front foot 90\textdegree out,
    back foot in 60\textdegree. 
  \item Turn the trunk to face away from wall.
  \item Focus on turning pelvis to face evenly perpendicular to mat. 
  \item With left hand, grab left thigh and pull/rotate it forward, to
    help turn the pelvis. 
  \item Repeat on other side.
  \end{itemize}
\item \newpose{\utka{}} (arms first) \PC{3}{8}
  \begin{itemize}
  \item Arms up straight in \urdnam{}.
  \item Try to keep elbows straight and bring palms as close together
    as possible. 
  \item Bend the knees until thighs are parallel to floor. Keep heels
    down. 
  \end{itemize}
\item \parsvo{} \PC{3}{9}
  \begin{itemize}
  \item Start with concave stage (use blocks if necessary) 
  \item Then head down
  \end{itemize}
\item \newpose{\badhastutt{}} \PC{3}{10}
  \begin{itemize}
  \item Start in \tad{}, feet apart.
  \item \badhasttad{}
  \item Exhale, stretch the trunk up, then forward, then down. 
  \item Come up, change the crossing of the arms, and repeat.
  \end{itemize}
\item \ardhal{}, to chair \PC{3}{11}
\item \paschi{} \PC{3}{12}
\item \newpose{setubandsarv{}} (cross bolsters) \PC{3}{13}
  \begin{itemize}
  \item Get two bolsters - a round and a flat.
  \item Put the round perpendicular across your mat
  \item Put the flat bolster lengthwise over the round bolster,
    forming a ``+'' 
  \item Lie down along top bolster so that head and shoulders come to
    floor.  
  \item Stretch legs straight, heels resting on floor.
  \item Arms out to the side.
  \end{itemize}
\item \sav{}  \PC{3}{14}
\end{poselist}


\chapter{Week 4}
\label{chap:week4}

\setcounter{week}{4}

\textbf{Themes}: 

\textbf{Reading}: 

\section{Sequence for Week 4}
\label{chap:seq4}

\begin{poselist}
\item \tad{} \PC{4}{1}
\item \urdbad{} \PC{4}{2}
\item \utttrik{} \PC{4}{3}
\item \viraii{} \PC{4}{4}
\item \uttparsva{} \PC{4}{5}
\item \newpose{\vim{}} \PC{4}{6}
  \begin{itemize}
  \item Left foot to wall
  \item Turn front foot 90\textdegree out, back foot 60\textdegree.
  \item Hands on waist
  \item Bend front leg to a 90\textdegree angle
  \item Keep trunk straight, perpendicular to floor.
  \item Keep back heel down.
  \item Straighten front leg, turn feet parallel, 
  \item Repeat on other side.
  \end{itemize}
\item \newpose{\virai{}} \PC{4}{7}
  \begin{itemize}
  \item Start with \vim{}
  \item Arms into \urdhast{}
  \item Experiment with starting with hands on waist, then knee bend,
    then \vim{} arms, versus starting with \utthastpad{} hands before
    bending leg.
  \end{itemize}
\item \utka{}  \PC{4}{8}
\item \parsvo{} \PC{4}{9}
\item \newpose{\utt} (full pose) \PC{4}{10}
  \begin{itemize}
  \item Get two blocks in case needed
  \item Feet apart
  \item Start with \badhasttad{}
  \item Then \badhastutt{}
  \item Finally, extend arms to floor, using blocks if needed. 
  \item Come up. Feet together, \urdhast{} hands, extend, bend, bring
    hands to floor or blocks.
  \item Bend at hips, keeping back extended. 
  \end{itemize}
\item \ardhal{} (from chair) \PC{4}{11}
\item \newpose{\ekapadsarv{}} (from \ardhal{} on chair) \PC{4}{12}
  \begin{itemize}
  \item From \ardhal{}, left right leg straight towards ceiling.
  \item Keep back straight.
  \item Bring right leg down, repeat with left leg.
  \end{itemize}
\item \paschi{} \PC{4}{13}
\item \setubandsarv{} \PC{4}{14}
\item \sav{} \PC{4}{15}
\end{poselist}


\chapter{Week 5}
\label{chap:week5}

\setcounter{week}{5}

\textbf{Themes}: Consolidation, \ams{}!, \sarv{}!

\textbf{Reading}: 

\section{Sequence for Week 5}
\label{chap:seq5}

\begin{poselist}
\item \tad{} \PC{5}{1}
\item \urdhast{} \PC{5}{2}
\item \urdbad{}  \PC{5}{3}
\item \nam{}  \PC{5}{4}
\item \urdnam{} from \urdhast{}  \PC{5}{5}
\item \newpose{\paschbadhast{}}  \PC{5}{6}
  \begin{itemize}
  \item Hands behind back, hold elbows.
  \end{itemize}
\item \newpose{\gomu{}} (arms only) \PC{5}{7}
  \begin{itemize}
  \item Get a belt, drape over right shoulder.
  \item Stand in \tad{}
  \item Upper arm first. Right arm. Bend elbow, hand reaching down
    back. Use left hand to gently push elbow back. Elbow pointing
    straight up. Keep head up, facing forward.
  \item Lower arm. Sweep left arm out and around to back. Bend elbow.  
  \item Can hands clasp? If not, grab belt with right hand, then left
    hand. See if can walk hands towards each other along belt.
  \item Head up, abdomen in.
  \end{itemize}
\item \newpose{\paschnama{}}  \PC{5}{8}
  \begin{itemize}
  \item Move shoulder blades towards each other and into back.
  \item Scooch hands up back.
  \item Can you rotate hands and arms so that thumbs come together?
  \end{itemize}
\item \vrk{} \PC{5}{9}
\item \utka{} \PC{5}{10}
\item \utthastpad{} \PC{5}{11}
\item \uttparsva{}  \PC{5}{12}
\item \utttrik{}  \PC{5}{13}
\item \viraii{} \PC{5}{14}
\item \uttparsva{}  \PC{5}{15}
\item \vim{}  \PC{5}{16}
\item \virai{} \PC{5}{17}
\item \newpose{\praspad{}} (full pose)  \PC{5}{18}
\item \newpose{\ams{}}  \PC{5}{19}
  \begin{itemize}
  \item Mat to wall, stand against wall. 
  \item \utt{}, hands to floor. Walk hands forward 4 feet, keeping
    heels at the wall.
  \item Hands shoulder-width apart. Feet in line with hands.
  \item Hands: weight balanced left hand vs right hand. For each hand,
    weight balanced evenly front-to-back, pinky-to-thumb. Palms open,
    fingers spread apart.
  \item Arms: elbows straight
  \item Lengthen spine, raise buttocks toward ceiling
  \item legs straight, knees open
  \item Come up on toes, left buttocks as high as possible.
  \item Now keep buttocks up high while lengthening calves and ankle
    to bring heels down towards floor.
  \end{itemize}
\item \newpose{\utt{}} (concave back)  \PC{5}{20}
  \begin{itemize}
  \item Get 2 blocks
  \item Feet hip width apart.
  \item use blocks if hands cannot rest comfortably on floor.  
  \item Knees straight
  \item Weight even on feet front to back.
  \item repeat with feet together.
  \end{itemize}
\item \newpose{\padang{}}  \PC{5}{21}
  \begin{itemize}
  \item Use belt if cannot grab toes.
  \item Start with concave back. Head looking up/forward.
  \item Then release into head down.
  \end{itemize}
\item \ardhal{} (with chair)  \PC{5}{22}
  \begin{itemize}
  \item Teach use of belt, as preparation for \sarv{}.
  \end{itemize}
\item \newpose{\ekapadsarv{}} (from \ardhal{})  \PC{5}{23}
\item \newpose{\sarv{}} (from \ardhal{})  \PC{5}{24}
  \begin{itemize}
  \item Staying in \ardhal{},
  \item Bring right foot up into \ekapadsarv{}.
  \item Bring left foot up to join right foot.
  \item Straighten back using hands.
  \item Bring hands higher on back (towards neck).
  \end{itemize}
\item \newpose{\hal{}}  \PC{5}{25}
  \begin{itemize}
  \item Bring legs down to chair, into \ardhal{}.
  \item Push chair away from head,
  \item Bring legs down, bringing feet to floor.
  \item Hips should be over head.
  \end{itemize}
\item \paschi{}  \PC{5}{27}
  \begin{itemize}
  \item (skipping \karn{})
  \end{itemize}
\item \sav{}  \PC{5}{18}
\end{poselist}



\chapter{Week 6}
\label{chap:week6}

\setcounter{week}{6}

\textbf{Themes}: \ardchand{}!

\textbf{Reading}: 

\section{Sequence for Week 6}
\label{chap:seq6}

\begin{poselist}
\item \tad{} \PC{6}{1}
\item \urdbad{}  \PC{6}{2}
\item \paschbadhast{}  \PC{6}{3}
\item \gomu{}  \PC{6}{4}
\item \paschbadhast{}  \PC{6}{5}
\item \vrk{}  \PC{6}{6}
\item \utka{}  \PC{6}{7}
\item \utthastpad  \PC{6}{8}
\item \parshastpad{}  \PC{6}{9}
\item \utttrik{}  \PC{6}{10}
\item \viraii  \PC{6}{11}
\item \uttparsva  \PC{6}{12}
\item \vim  \PC{6}{13}
\item \virai \PC{6}{14}
\item \newpose{\ardchand{}}  \PC{6}{15}
  \begin{itemize}
  \item Have a block available.
  \item \utttrik{}
  \item Bend right knee, place right hand on floor about a foot in
    front of right foot.
  \item Left hand on waist
  \item Bring left foot in a little towards right foot
  \item Raise left leg so parallel to floor
  \item Extend right leg
  \item If cannot extend right leg with hand on floor, put right hand
    on block.
  \item Can you raise your left arm towards ceiling?
  \item Repeat on other side.
  \end{itemize}
\item \newpose{\parsvo{}} (full pose) \PC{6}{16}
\item \ams{}  \PC{6}{17}
\item \utt{} (concave back)  \PC{6}{18,19}
  \begin{itemize}
  \item Use blocks if neede
  \item First with feet apart, then repeat with feet together.
  \end{itemize}
\item \padang{}  \PC{6}{20}
\item \ardhal{}  \PC{6}{21}
\item \ekapadsarv{}  \PC{6}{22}
\item \sarv{}  \PC{6}{23}
\item \hal{}  \PC{6}{24}
\item \paschi{}  \PC{6}{26}
  \begin{itemize}
  \item skipping \karn{}
  \end{itemize}
\item \sav  \PC{6}{27}
\end{poselist}


\bibliography{yoga}
\bibliographystyle{alpha}


\end{document}
