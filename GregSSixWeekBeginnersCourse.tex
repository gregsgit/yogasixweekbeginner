\documentclass[letter, oneside]{book}

% \usepackage{layout}

\usepackage{amsmath}
\usepackage{capt-of}
\usepackage{framed}
\usepackage{hyperref}
% \usepackage[showframe]{geometry}
\usepackage[pdftex]{graphicx}
\usepackage{enumitem}
\usepackage{marginnote}
\usepackage{textcomp}   % for degree symbol
\usepackage{titlesec}
\usepackage{todonotes}
\usepackage{wrapfig}
\usepackage{color}
\usepackage{xifthen}
\usepackage{xstring}


% Pose shortcodes
\newcommand{\apose}[1]{\emph{#1}}
%
\newcommand{\ams}{\apose{Adho Mukha Svanasana}}
\newcommand{\ardhal}{\apose{Ardha Halasana}}
\newcommand{\ardchand}{\apose{Ardha Chandrasana}}
\newcommand{\badd}{\apose{Baddhanguliyasana}}
\newcommand{\badhasttad}{\apose{Baddha Hasta Tadasana}}
\newcommand{\badhastutt}{\apose{Baddha Hasta Uttanasana}}
\newcommand{\dand}{\apose{Dandasana}}
\newcommand{\ekapadsarv}{\apose{Eka Pada Sarvangasana}}
\newcommand{\gomu}{\apose{Gomukasana}}
\newcommand{\hal}{\apose{Halasana}}
\newcommand{\karn}{\apose{Karnapidasana}}
\newcommand{\nam}{\apose{Namaskarasana}}
\newcommand{\padadand}{\apose{Padanghusta Dandasana}}
\newcommand{\padang}{\apose{Padangusthasana}}
\newcommand{\parshastpad}{\apose{Parsva Hasta Padasana}}
\newcommand{\parsvo}{\apose{Parsvottanasana}}
\newcommand{\parsvoconc}{\apose{Parsvottanasana (concave)}}
\newcommand{\parsvostand}{\apose{Parsvottanasana (standing)}}
\newcommand{\paschi}{\apose{Paschimottanasana}}
\newcommand{\paschbadhast}{\apose{Paschima Baddha Hastasana}}
\newcommand{\paschnama}{\apose{Paschima Namaskarasana}}
\newcommand{\praspad}{\apose{Prasarita Padottanasana}}
\newcommand{\praspadconc}{\apose{Prasarita Padottanasana} (concave)}
\newcommand{\sarv}{\apose{Salamba Sarvangasana}}
\newcommand{\sav}{\apose{Savasana}}
\newcommand{\setubandsarv}{\apose{Setu Bandha Sarvangasana}}
\newcommand{\tad}{\apose{Tadasana}}
\newcommand{\urdbad}{\apose{Urdhva Baddhanguliyasana}}
\newcommand{\urdhast}{\apose{Urdhva Hastasana}}
\newcommand{\urdhastdand}{\apose{Urdhva Hasta Dandasana}}
\newcommand{\urdnam}{\apose{Urdhva Namaskarasana}}
\newcommand{\utka}{\apose{Utkatasana}}
\newcommand{\utt}{\apose{Uttanasana}}
\newcommand{\utthastpad}{\apose{Utthita Hasta Padasana}}
\newcommand{\uttparsva}{\apose{Utthita Parsvakonasana}}
\newcommand{\utttrik}{\apose{Utthita Trikonasana}}
\newcommand{\vim}{\apose{Vimanasana}}
\newcommand{\virai}{\apose{Virabhdrasana I}}
\newcommand{\viraii}{\apose{Virabhadrasana II}}
\newcommand{\vrk}{\apose{Vrksasana}}

\makeatletter
\renewcommand{\fnum@figure}{\thefigure}
\makeatother

% 1st arg is filename, 2nd arg (optional) is caption.
% if 2nd arg empty, use filename as caption
\newcommand{\poseFig}[1]{
  \begin{wrapfigure}{l}{0.8in}
    \includegraphics[height=0.8in, width=0.8in, keepaspectratio]{Figures/{"#1"}.jpg}
  \end{wrapfigure}
}

%  \begin{minipage}{1.0in}
%    \captionof{figure}{\ifthenelse{\isempty{#2}}{#1}{#2}}
%    \label{fig:\theweek.#1}
  % \vspace{1ex}
%  \end{minipage}


\newcommand{\amsFig}{\poseFig{Adho Mukha Svanasana}}
\newcommand{\ardhalFig}{\poseFig{Ardha Halasana}}
\newcommand{\ardchandFig}{\poseFig{Ardha Chandrasana}}
\newcommand{\baddFig}{\poseFig{Baddhanguliyasana}}
\newcommand{\badhasttadFig}{\poseFig{Baddha Hasta Tadasana}}
\newcommand{\badhastuttFig}{\poseFig{Baddha Hasta Uttanasana}}
\newcommand{\dandFig}{\poseFig{Dandasana}}
\newcommand{\ekapadsarvFig}{\poseFig{Eka Pada Sarvangasana}}
\newcommand{\gomuFig}{\poseFig{Gomukasana}}
\newcommand{\halFig}{\poseFig{Halasana}}
\newcommand{\karnFig}{\poseFig{Karnapidasana}}
\newcommand{\namFig}{\poseFig{Namaskarasana}}
\newcommand{\padadandFig}{\poseFig{Padanghusta Dandasana}}
\newcommand{\padangFig}{\poseFig{Padangusthasana}}
\newcommand{\parshastpadFig}{\poseFig{Parsva Hasta Padasana}}
\newcommand{\parsvoFig}{\poseFig{Parsvottanasana}}
\newcommand{\parsvoConcFig}{\poseFig{ParsvottanasanaConcave}}
\newcommand{\parsvoStandFig}{\poseFig{ParsvottanasanaStanding}}
\newcommand{\paschiFig}{\poseFig{Paschimottanasana}}
\newcommand{\paschbadhastFig}{\poseFig{Paschima Baddha Hastasana}}
\newcommand{\paschnamaFig}{\poseFig{Paschima Namaskarasana}}
\newcommand{\praspadFig}{\poseFig{Prasarita Padottanasana}}
\newcommand{\praspadconcFig}{\poseFig{PrasaritaPadottanasanaConcave}}
\newcommand{\sarvFig}{\poseFig{Salamba Sarvangasana}}
\newcommand{\savFig}{\poseFig{Savasana}}
\newcommand{\setubandsarvFig}{\poseFig{Setu Bandha Sarvangasana}}
\newcommand{\tadFig}{\poseFig{Tadasana}}
\newcommand{\urdbadFig}{\poseFig{Urdhva Baddhanguliyasana}{}}
\newcommand{\urdhastFig}{\poseFig{Urdhva Hastasana}}
\newcommand{\urdhastdandFig}{\poseFig{Urdhva Hasta Dandasana}}
\newcommand{\urdnamFig}{\poseFig{Urdhva Namaskarasana}}
\newcommand{\utkaFig}{\poseFig{Utkatasana}}
\newcommand{\uttFig}{\poseFig{Uttanasana}}
\newcommand{\utthastpadFig}{\poseFig{Utthita Hasta Padasana}}
\newcommand{\uttparsvaFig}{\poseFig{Utthita Parsvakonasana}}
\newcommand{\utttrikFig}{\poseFig{Utthita Trikonasana}}
\newcommand{\vimFig}{\poseFig{Vimanasana}}
\newcommand{\viraiFig}{\poseFig{Virabhdrasana I}}
\newcommand{\viraiiFig}{\poseFig{Virabhadrasana II}}
\newcommand{\vrkFig}{\poseFig{Vrksasana}}


\newcommand{\poseMarginFig}[1]{
  \marginpar{\includegraphics[height=0.7in, width=0.7in, keepaspectratio]{Figures/{"#1"}.jpg}}
}

\newcommand{\amsMarginFig}{\poseMarginFig{Adho Mukha Svanasana}}
\newcommand{\ardhalMarginFig}{\poseMarginFig{Ardha Halasana}}
\newcommand{\ardchandMarginFig}{\poseMarginFig{Ardha Chandrasana}}
\newcommand{\baddMarginFig}{\poseMarginFig{Baddhanguliyasana}}
\newcommand{\badhasttadMarginFig}{\poseMarginFig{Baddha Hasta Tadasana}}
\newcommand{\badhastuttMarginFig}{\poseMarginFig{Baddha Hasta Uttanasana}}
\newcommand{\dandMarginFig}{\poseMarginFig{Dandasana}}
\newcommand{\ekapadsarvMarginFig}{\poseMarginFig{Eka Pada Sarvangasana}}
\newcommand{\gomuMarginFig}{\poseMarginFig{Gomukasana}}
\newcommand{\halMarginFig}{\poseMarginFig{Halasana}}
\newcommand{\karnMarginFig}{\poseMarginFig{Karnapidasana}}
\newcommand{\namMarginFig}{\poseMarginFig{Namaskarasana}}
\newcommand{\padadandMarginFig}{\poseMarginFig{Padanghusta Dandasana}}
\newcommand{\padangMarginFig}{\poseMarginFig{Padangusthasana}}
\newcommand{\parshastpadMarginFig}{\poseMarginFig{Parsva Hasta Padasana}}
\newcommand{\parsvoMarginFig}{\poseMarginFig{Parsvottanasana}}
\newcommand{\parsvoConcMarginFig}{\poseMarginFig{ParsvottanasanaConcave}}
\newcommand{\parsvoStandMarginFig}{\poseMarginFig{ParsvottanasanaStanding}}
\newcommand{\paschiMarginFig}{\poseMarginFig{Paschimottanasana}}
\newcommand{\paschbadhastMarginFig}{\poseMarginFig{Paschima Baddha Hastasana}}
\newcommand{\paschnamaMarginFig}{\poseMarginFig{Paschima Namaskarasana}}
\newcommand{\praspadMarginFig}{\poseMarginFig{Prasarita Padottanasana}}
\newcommand{\praspadconcMarginFig}{\poseMarginFig{PrasaritaPadottanasanaConcave}}
\newcommand{\sarvMarginFig}{\poseMarginFig{Salamba Sarvangasana}}
\newcommand{\savMarginFig}{\poseMarginFig{Savasana}}
\newcommand{\setubandsarvMarginFig}{\poseMarginFig{Setu Bandha Sarvangasana}}
\newcommand{\tadMarginFig}{\poseMarginFig{Tadasana}}
\newcommand{\urdbadMarginFig}{\poseMarginFig{Urdhva Baddhanguliyasana}}
\newcommand{\urdhastMarginFig}{\poseMarginFig{Urdhva Hastasana}}
\newcommand{\urdhastdandMarginFig}{\poseMarginFig{Urdhva Hasta Dandasana}}
\newcommand{\urdnamMarginFig}{\poseMarginFig{Urdhva Namaskarasana}}
\newcommand{\utkaMarginFig}{\poseMarginFig{Utkatasana}}
\newcommand{\uttMarginFig}{\poseMarginFig{Uttanasana}}
\newcommand{\utthastpadMarginFig}{\poseMarginFig{Utthita Hasta Padasana}}
\newcommand{\uttparsvaMarginFig}{\poseMarginFig{Utthita Parsvakonasana}}
\newcommand{\utttrikMarginFig}{\poseMarginFig{Utthita Trikonasana}}
\newcommand{\vimMarginFig}{\poseMarginFig{Vimanasana}}
\newcommand{\viraiMarginFig}{\poseMarginFig{Virabhdrasana I}}
\newcommand{\viraiiMarginFig}{\poseMarginFig{Virabhadrasana II}}
\newcommand{\vrkMarginFig}{\poseMarginFig{Vrksasana}}


% \PC{week}{pose #}
\newcommand{\PC}[2]{{\normalfont\normalsize \hfill(PC w{#1}.p{#2})}}

\newcommand{\posenote}[1]{{\normalfont\normalsize \hfill(#1)}}

% \term{xxx} - introduce a standard Iyengar word or phrase
\newcommand{\term}[1]{``{#1}''}

% \newpose{xxx} - first time this pose has been used in course
\newcommand{\newpose}[1]{{\color{blue}{#1}}}

\newcounter{week}
\newcounter{pose}

\newcommand{\week}[1]
{ \IfDecimal{#1}{\setcounter{week}{\integerpart}}{fooey}
  \setcounter{pose}{1}
  \chapter{Week {#1}}}

% a week is a chapter
\titleformat{\chapter}[frame]
{} % format
{} % label
{8pt} % sep
{\LARGE\bfseries\filcenter} % before-code
{} % after-code

% a pose is a subsection
\newcommand{\pose}{\subsection}
\titleformat{\subsection}[hang]
{\normalfont} % format
{\fbox{w\theweek.p\thepose}} % label
{0.5em} % sep
{\Large\bfseries\refstepcounter{pose}} % before-code
{} % after-code

\titlespacing{\subsection}
{-5.0ex} % left
{1.5ex plus .1ex minus .2ex} % before-sep
{1pc} % after-sep
% [] % right-sep


\newcommand{\linknote}[1]{\todo[color=green!40]
  {\small Linking: #1}}
\newcommand{\actionnote}[1]{\todo[color=cyan!40]
  {\small Primary actions: #1}}
\newcommand{\termnote}[1]{\todo[color=violet!40]
  {\small Terminology: #1}}
\newcommand{\themenote}[1]{\todo[color=magenta!40]
  {\small Theme: #1}}

\setlength{\parindent}{0cm}
\setlength{\parskip}{1ex}


\title{Six Week Iyengar Yoga Course for Beginners}
\author{Greg Sullivan \href{mailto:gregs@sulliwood.org}{gregs@sulliwood.org} \\
For Heart of Iyengar Yoga Teacher Training course with Peentz Dubble
}
\date{March 11, 2017}

\begin{document}

% \layout

\maketitle

\tableofcontents

\chapter*{Overview / Goals}
\label{chap:overview}

\addcontentsline{toc}{chapter}{Overview/Goals}

The overriding goal of a class for beginning students is to instill an
enjoyment and appreciation of yoga practice. The goal is to give
beginning students a taste for the benefits of yoga practice, and to,
as much as possible, give these beginning students tools that they can
use to safely and joyfully continue to practice yoga. As Geeta Iyengar
writes in the preface of ``Yoga in Action: A Preliminary
Course''\cite{GeetaPC}, 

\begin{quote}
  \emph{Yoga in Action for Beginners is not the end, but the beginning of
  yoga. It is for the practitioner to ignite the hidden force of yoga
  from within, so that it throws the Light on the path of the yogic
  journey.}
\end{quote}

Of course, Preliminary Course gives a 28-week syllabus, and for this
assignment we are creating only a 6-week course.

The immediate goals of the first few classes are to help students
familiarize themselves with their bodies, and also with the
terminology we use to describe motions and poses in Iyengar Yoga
classes.

Yoga, and in particular asana practice, is not about how you look. If
your pose is not geometrically perfect, you are not a failure. One of
the lessons that yoga teaches is vairagya, detachment. Along with
detachment is non-judgement. Judging yourself, in an asana, as ``good''
or ``bad'' is not helpful. The question is ``what is the goal?'' How do
you strive towards the goal? In this way, asana practice is a
microcosm of ``life practice''. The concept of ``bringing yoga off the
mat'' is important

I have mostly followed the sequences given for the first six weeks in
Geeta Iyengar's \emph{Preliminary Course}\cite{GeetaPC}, abbreviated
``PC''. For each pose given, I indicate whether it was in that week's
sequence in PC, and, if so, which number in the sequence in PC.
For example, if the fourth pose in week 2 is \nam{}, and it is also
the fourth pose in the week 2 syllabus in PC, I will write:

\fbox{w2.p4} \nam{} \PC{2}{4}

\section*{Formatting Notes}
\label{chap:formatting}

\addcontentsline{toc}{section}{Formatting Notes}

Pose names are italicized, as in \apose{this is a pose name}.

A new pose (not introduced earlier) will be both italicized and
colored blue, as in \newpose{\apose{this is a new pose}}.

Images of poses are taken from \emph{Preliminary Course}, as I do not
yet have images of myself in the poses.

Margin notes are in variously colored boxes. \actionnote{primary
  action notes look like this.}\linknote{Notes on linking actions look
like this.}\termnote{Notes on terminology look like
this.}\themenote{and theme notes look like this.}


\week{1}
\label{week:1}

\textbf{Themes}: What is yoga? Your body in space.

\textbf{Reading}: Something that relates asana to the rest of one's
life; how looking for alignment and balance in asana can be brought
``off the mat'' to bring alignment and balance in the rest of one's
life.

\section{Sequence with Instructions for Week 1}
\label{seq:1}

\pose{\newpose{\sav{}} \posenote{not given as first pose in PC}}

\savFig{}
Have several blankets available if needed.

Lie flat; legs together, toes pointing towards ceiling. Knees pointing
to ceiling. Hips level. Feel what parts of your body are touching the
ground. Is there equal weight on both heels? Both buttocks? Shoulder
blades? Try un-balancing the weight, to one side or the other; then
re-balance. 

%\savMarginFig{}

Now bring your arms over your head. Can you reach all the way to the
ground? Is one arm and shoulder more flexible than the other?  Use
blankets if need support. How close are hands; can you bring them
closer together? Rotate your arms so that your palms face the
ceiling. Now reverse the rotation; which causes the shoulder blades to
separate? Try rolling the upper arms (nearer the shoulders)
``inward''\termnote{Inward rotation} so that palms turn toward floor.
\actionnote{Turning arms ``inward''.}  

Now keep upper arms turned in,
but make palms parallel.

Now try to increase the distance between top of shoulders and ears
(i.e. lower shoulders) while keeping arms straight above head. Try the
reverse -- shrugging shoulders up towards ears; then reverse.
\actionnote{Lower shoulders away from ears.}

\newpage
\pose{\newpose{\tad{}}, against a wall \PC{1}{1}}

% \tadMarginFig{}
\tadFig{}

Now we are going to attempt to replicate the pose we just did
on the floor, with the help of gravity, while standing.
\linknote{Same shape between \tad{} and \sav{}}

In \sav{}, the floor was a useful reference point -- we
could tell if our hips were even, or our shoulder blades were
even, by feeling the contact with the floor.

Stand with your back lightly touching the wall; your heels as
close to the wall as possible.

Have your feet together.

Rotate your outer thighs towards each other.\actionnote{Rotate thighs
  towards each other.}

Is your weight balanced on your feet, left vs. right? Try
shifting all your weight to your left foot; now the right foot.

Is your weight evenly balanced forward and backward? That is, is the
weight on the balls of your feet the same as the weight on your heels?
Try moving forward, putting more weight on the balls of the feet. Now
try lifting the toes, putting all the weight on the heels. Now back to
even.

Consider your left foot. Is the weight evenly balanced on the ``4
corners''\termnote{4 corners of foot.} of the foot? Same for right
foot.

Abdomen in, lift chest. Shoulders slightly back.\actionnote{Lift chest.}

Shoulder blades slightly towards each other.

Lower your shoulders away from your ears.\linknote{Shoulders lowered
  as in \sav{}.}

Remember what was touching the floor in Savasana? Lightly touching the
wall, see if you can replicate the touchpoints from Savasana. Your
heels, buttocks', shoulder blades, and back of head.


\pose{\newpose{\urdhast{}} \PC{1}{2}}
%\urdhastMarginFig{}

\urdhastFig{}
Now raise your arms above your head. This is called \urdhast{}
(arms). Recall where they were when you were on your back; can you
position them the same way without gravity? Turn your arms ``inward''
again, so that palms turn towards wall. Now keep upper arms turned,
but make palms parallel.\linknote{\urdhast{} arms in \sav{}.}
Can you bring your arms as close to the wall as they were to the floor
in \sav{}?\actionnote{Rotate arms inward.}

Lower shoulders away from ears.\actionnote{Lower shoulders.}

\newpage

\mbox{ }

\pose{\newpose{\urdbad{}} \PC{1}{3}}

%\urdbadMarginFig{}

\urdbadFig{}
Bring your arms down to pointing straight in front of you.


Clasp your hands. Note which thumb is on top.

Separate your wrists and rotate the thumb sides down. This is \badd{}.

Raise your arms, as you did in \urdhast{}, keeping your hands in
\badd{}; Straighten your arms, especially the elbows. Pull elbows
towards each other.\actionnote{Elbows straight, towards each other.}

Lower your arms, change the interlock of your fingers so that other
thumb is on top, repeat.

\pose{\newpose{\nam{}} \PC{1}{4}}
% \namMarginFig{}

\namFig{}
In \tad{}, bring hands together as in prayer.

Can you lower your shoulders, raise your chest, bring your shoulders
slightly back?\actionnote{Shoulders back slightly, open chest.}

Without disturbing any elements of \tad{}?

\mbox{ }

\mbox{ }

\pose{\newpose{\urdnam{}} from \newline \urdhast{} \PC{1}{5}}
% \urdnamMarginFig{}

\urdnamFig{}
How close can you bring your palms while maintaining the integrity of
the pose? What is preventing you from bringing your palms together?

\mbox{ }

\mbox{ }

\begin{framed}

\mbox{ }

\textbf{Interlude}. So far everything has been perfectly straight and
balanced, left to right, top to bottom. Now we're going to branch out.

\mbox{ }

\end{framed}

\newpage 

\pose{\newpose{Half \utt{}} (not in PC)}

Bring mats to wall.

Stand legs-distance from wall. \themenote{Body in space; how long is
  your torso?}  Lean over so that torso is parallel to floor, forming
right angle; hands touching wall. Above waist is \urdhast{} (rotated
90\textdegree). Below waist is \tad{}.  \linknote{\urdhast{} hands,
  \tad{} legs.}

\pose{\newpose{\utthastpad{}} \PC{1}{6}}
% \utthastpadMarginFig{}

\utthastpadFig{}
Left foot to wall.

Separate feet 4-5 feet apart. Arms out, parallel to floor. Even
balance front-to-back, left to right.

Lift trunk and chest. Lower shoulders, as in \tad{}.\linknote{\tad{}
  shoulders.} 

\mbox{ }

\mbox{ }

\mbox{ }


\pose{\newpose{\parshastpad{}} \PC{1}{7}}
% \parshastpadMarginFig{}

\parshastpadFig{}
Turn right leg out 90\textdegree. Right heel in line with center of
left foot (both feet centered on mat). Toes pointing 90\textdegree,
left knee pointing 90\textdegree. Turn left foot in
slightly. Everything else unchanged. Hips still parallel to long edge
of mat; torso facing forward; face facing forward. Are hips even (same
height)?

The only thing about your pose that should have changed is your right
leg.\themenote{Keeping some elements of pose/body the same, while
  changing others.}

Are your arms getting tired?\themenote{Body awareness; how heavy are
  your arms?}

\newpage

\pose{\newpose{\utttrik{}} \PC{1}{8}}
% \utttrikMarginFig{}

\utttrikFig{}
Extend the right arm out as you bend at the waist and bring the right
arm down to the shin. Place your left arm on your left waist. Keep
your torso perpendicular to the ground, requiring rotation
right-to-left of the torso. Now bring left arm up, pointing to
ceiling; two arms going in opposite directions. Lower shoulders
(increase distance between ears and shoulders). Are hips perpendicular
to the wall?

Are both sides of torso same length? Or did lower torso shorten? How
to lengthen lower torso? What does pelvis want to do - rotate left
side towards floor? What muscle actions are required to counter these
tendencies? \themenote{Body awareness}.

Come up, turn feet parallel.

Optional variation: Can try parallel to wall, so that wall gives
feedback to buttocks, shoulder blades, and head about alignment.

\begin{framed}
(repeat \utthastpad{}, \parshastpad{}, \utttrik{} on opposite side
(left foot to the wall).
\end{framed}


\pose{\newpose{\parsvoconc{}} \PC{1}{9}}
% \parsvoConcMarginFig{}

\parsvoConcFig{}
Grab two blocks.

Left foot to wall, Parsva Hasta Padasana. Now turn back foot in more,
and rotate hips to parallel wall. Hands to hips. Balance pelvis -
front-to-back and side-to-side. Chest facing front of mat, head facing
straight ahead.

Extend back, make concave, look up.

Now bend so that torso is parallel to floor (recall half-uttanasana).
\linknote{link to half \utt{}.}

Put blocks at whatever height allows you to keep concave back.

Repeat with right foot to wall.


\pose{\newpose{\praspadconc{}} \PC{1}{10}}
% \praspadconcMarginFig{}

\praspadconcFig{}
Wide separation of feet. Feet parallel to each other; even
w.r.t. distance from edge of mat.\themenote{Organizing orientation
  using mat.}

Bend at waist so that torso is parallel to floor. Back
concave\linknote{Concave back as in \parsvoconc{}}. Hands directly
below shoulders on floor. Use blocks if needed to get torso parallel
to floor.

Hips over (in line with) feet. Weight even front-to-back on
feet. Feet flat, weight even inner-to-outer on feet.

\pose{\newpose{\dand{}}  \PC{1}{11}}
% \dandMarginFig{}

\dandFig{}
May want to sit on some blankets. (Show how to fold two blankets.)

Sitting on floor. Knees pointing up, feet together -- same as
\sarv{}\linknote{\sarv{} legs.}

Abdomen perpendicular to floor, chest lifted -- same as \tad{}.

This is same body position as half \utt{}.\linknote{Rotation of half
  \utt{}.}

Put your hands on floor next to hips and press down to elongate the
back.\actionnote{Use hands to elongate back.}


\pose{\newpose{\urdhastdand{}} \PC{1}{12}}
% \urdhastdandMarginFig{}

\urdhastdandFig{}
As we did going from \tad{} to \urdhast{} arms, raise arms to
\urdhast{}.

Arms straight, palms parallel. Arms pointed straight at ceiling. 

Try with and without blankets. Can you keep pelvis level without
blankets? 

\mbox{ } 

\mbox{ } 

\pose{\newpose{\padadand{}} \PC{1}{13}}
% \padadandMarginFig{}

\padadandFig{}
Use a belt for integrity of back.

Concave back. Shoulders as in \tad{}.\actionnote{Concave back,
  shoulders down and even with torso.}


\mbox{ } 

\mbox{ } 

\mbox{ } 

\mbox{ } 

\newpage


\pose{\newpose{\paschi{}} \PC{1}{14}}
% \paschiMarginFig{}

\paschiFig{}
Grab a bolster and a few more blankets.

This is a good time to remind students to not be attached to the
perfect form/geometry of the asana, but to consider the
goals.\themenote{Not about achieving a particular geometry.}

Sit on a few blankets.

Start in \padadand{}. 

Belt around feet. Focus initially on moving chest towards toes. Back
remains straight until have achieved maximum rotation in pelvis.

Place bolster and blankets so that you can come to rest on blankets or
bolster. 

\pose{\sav{}}
\savMarginFig{}

% \section{Sequence, in Images, for Week 1}
% \label{seqimags:1}

% \begin{tabular}{|c|c|c|c|}
% \savFig{} &
% \tadFig{} &
% \urdhastFig{} &
% \urdbadFig{} \\ \hline
% \namFig{} &
% \urdnamFig{} & 
% \utthastpadFig{} &
% \parshastpadFig{} \\ \hline
% \utttrikFig{} &
% \parsvoFig &
% \praspadFig &
% \dandFig \\ \hline
% \urdhastdandFig &
% \padadandFig &
% \paschiFig &
% \savFig 
% \end{tabular}

\week{2}
\label{week:2}

\textbf{Themes}: Bent leg poses (\viraii{}, \uttparsva{}). Initial
taste of inversion (\ardhal{} on chair). Also, consolidation of asanas
from first week, and some strengthening. 

\textbf{Reading}:

\section{Sequence with Instructions for Week 2}
\label{seq:2}

\pose{\tad{} \PC{2}{1}}

\tadFig{}
Feet together. Legs straight. Rotate thighs towards each other.

% \tadMarginFig{}
Abdomen in. Chest raised. Shoulders in line with front-to-back center
of body.

Upper arms rotate out. Shoulder blades slightly together. Hands
parallel to hips.

Shoulders lowered away from ears.

Gaze straight ahead.

Balanced front-to-back, side-to-side.

\pose{\urdhast{} \PC{2}{2}}
% \urdhastMarginFig{}

\urdhastFig{}
Rotate upper arms inward (forward-facing parts of arms go towards each
other).\termnote{Rotating arms inward.}

\mbox{ }

\mbox{ }

\mbox{ }

\mbox{ }

\mbox{ }


\pose{\urdbad{} \PC{2}{3}}
\urdbadMarginFig{}

Repeat for each thumb on top. 

\pose{\nam{} \PC{2}{4}}
\namMarginFig{}

\pose{\urdnam{} \PC{2}{5}}
\urdnamMarginFig{}

\pose{\utthastpad{} \PC{2}{6}}
\utthastpadMarginFig{}

\pose{\parshastpad{} \PC{2}{7}}
\parshastpadMarginFig{}

\pose{\utttrik{} \PC{2}{8}}
% \utttrikMarginFig{}

\utttrikFig{}
Finer points: Weight on back leg (at least even with weight on front
leg). Action to length lower abdomen. 

May try against wall.

\mbox{ }

\mbox{ }

\mbox{ }

\mbox{ }

\pose{\newpose{\viraii{}} \PC{2}{9}}
%\viraiiMarginFig{}

\viraiiFig{}
Start in \parshastpad{} - as in start of \utttrik{}.\linknote{Related
  to \utttrik{}.}

(do pose with hands on waist first, then with arms extended)

Keeping everything stable, bend right knee to 90\textdegree
\themenote{Visually verify 90\textdegree.}

Is torso centered over hips, between feet?

Is weight solid on back foot?

Extend inner thigh; pull back outer thigh.\termnote{Extend inner
  thigh, contract outer thigh.}

Look over front arm.

Repeat on other side

\pose{\newpose{\uttparsva{}} \PC{2}{10}}
% \uttparsvaMarginFig{}

\uttparsvaFig{}
Have a block handy, about 4 feet from wall.

Again, start in \parshastpad{}

Again, bend right knee to 90\textdegree. Same as
\viraii{}.\linknote{Connect to \viraii{}.}

Left hand on waist.

Bring right hand to floor, or block, keeping chest facing wall in
front of you (perpendicular to floor).

Rotate abdomen.

Raise left arm towards ceiling.

Rotate left arm as in \utthastpad{}.\actionnote{Arm rotation.}

See if you can bring arm down alongside ear.
If not, what is getting in the way?

Are your quadriceps getting tired?\themenote{Strength.}

\pose{\parsvoconc{} (standing, then concave back) \PC{2}{11}}
\parsvoConcMarginFig{}

Use two blocks to keep back concave.

Look forward.


\pose{\newpose{\parsvo{}} \PC{2}{12}}
% \parsvoMarginFig{}

\parsvoFig{}
Bend at waist, sternum directly over front knee, put hands on
blocks (at highest setting), keeping back concave.

Now walk blocks forward, extending back.

Finally, bring head down over knee.

Process of bringing head down is similar to
\paschi{}\linknote{Bringing head down as in \paschi{}.} 
Extend back first. Aim chest towards shin.

Finally allow back and neck to relax and bring head down. 

\newpage

\pose{\praspad{} \PC{2}{13}}
% \praspadMarginFig{}

\praspadFig{}
Start with concave back.

Check orientation using mat.

Check balance is even front-to-back. Hips are over center-line running
between feet.

\mbox{ }

\mbox{ }

\pose{\dand{} \PC{2}{14}}
\dandMarginFig{}

\pose{\urdhastdand{} \PC{2}{15}}
\urdhastdandMarginFig{}

\pose{\padadand{} \PC{2}{16}}
\padadandMarginFig{}

Use belt if needed.

On blankets if needed.

\pose{\newpose{\ardhal{}} to chair \PC{2}{17}}
% \ardhalMarginFig{}

\ardhalFig{}
Get 4 blankets, a block, and a chair

3 blankets - set up as for Sarvangasana.

Chair at head, 1 blanket on chair. Block for tailbone.

Shoulders 2 in.s from edge of blankets.

Swing up, bringing legs to chair.

Goals: hips over head, legs parallel to floor.

Arm positions for Sarvangasna (w/o belts). Elbows down and in. Hands
on mid-to-upper back. Use hands to lift hips and straighten
back.\actionnote{Back straight and perpendicular to floor.}



\pose{\paschi{} \PC{2}{18}}
\paschiMarginFig{}

\pose{\sav{} \PC{2}{19}}
  \savMarginFig{}

% \newpage
% \section{Sequence in Images for Week 2}
% \label{seqimags:2}

% \begin{tabular}{|c|c|c|c|}
% \tadFig &
% \urdhastFig &
% \urdbadFig &
% \namFig{} \\ \hline
% \urdnamFig{} &
% \utthastpadFig{} &
% \parshastpadFig{} &
% \utttrikFig{} \\ \hline
% \viraiiFig{} &
% \uttparsvaFig{} &
% \parsvoFig{} &
% \parsvoFig{} \\ \hline
% \praspadFig{} &
% \dandFig{} &
% \urdhastdandFig{} &
% \padadandFig{} \\ \hline
% \paschiFig{} &
% \savFig{} & & 
% \end{tabular}

\week{3}
\label{week:3}

\textbf{Themes}: Balancing (e.g. \vrk{}). Towards \sarv{} (\ardhal{}
on chair, \setubandsarv{})

\textbf{Reading}: 

\section{Sequence with Instructions for Week 3}
\label{seq:3}

\pose{\tad{} \PC{3}{1}}
\tadMarginFig{}

Focus on sequence feet-to-head.

Engage muscles, then relax a bit.


\pose{\urdbad{} \PC{3}{2}}
\urdbadMarginFig{}


\pose{\newpose{\vrk{}} \PC{3}{3}}
% \vrkMarginFig{}

\vrkFig{}
Stand with back to wall, far enough away so that can reach back to
touch wall for balance.

Bend the right knee, keeping left leg in Tadasana, grab right foot
with right hand.

Place right foot high on inside of left thigh.

Push thigh out against foot, foot in against thigh.\themenote{Opposing
  actions.} 

Push knee back - lengthen inner right thigh, pull in outer right
thigh (also oppositional actions). Same thigh actions as \viraii{} and
\uttparsva{}\linknote{Knee and thigh actions of \viraii{} and \uttparsva{}}

Can you get your knee parallel with wall (perpendicular to gaze)?

Balance free of wall.\themenote{Balance}

Can you raise your left hand straight up?

Now your right hand?

Can you bring your hands together into \urdnam{}?

Lower arms, lower right leg, repeat with left leg.

\pose{\utttrik{} \PC{3}{4}}
\utttrikMarginFig{}

\pose{\viraii{} \PC{3}{5}}
\viraiiMarginFig{}

\pose{\uttparsva{} \PC{3}{6}}
\uttparsvaMarginFig{}

\pose{\newpose{\vim{}} \PC{4}{6}}
% \vimMarginFig{}

\vimFig{}
Stand left foot to wall. Turn front foot 90\textdegree out, back
foot in 60\textdegree.\linknote{As in \parsvo{}, coming up}

Turn the trunk to face away from wall.

Focus on turning pelvis to face evenly perpendicular to
mat.\actionnote{Turning pelvis perpendicular to forward leg}

With left hand, grab left thigh and pull/rotate it forward, to
help turn the pelvis.

Torso perpendicular to floor.

Arms out to sides, like an airplane.

Repeat on other side.

\pose{\newpose{\utka{}} (arms first) \PC{3}{8}}
% \utkaMarginFig{}

\utkaFig{}
Arms up straight in \urdnam{}.

Try to keep elbows straight and bring palms as close together as
possible.

Bend the knees until thighs are parallel to floor. Keep heels
down. If absolutely cannot keep heels down, put a blanket under them,
as thin as possible.

This is a bit of a balance (forward-backward).\themenote{Balance}


\pose{\parsvo{} \PC{3}{9}}
\parsvoMarginFig{}

Start with concave stage (use blocks if necessary)

Then head down.


\pose{\newpose{\badhastutt{}} \PC{3}{10}}
% \badhastuttMarginFig{}

\badhastuttFig{}
Start in \tad{}, feet apart.

\badhasttad{} arms. Remember which arm is on top.

Exhale, stretch the trunk up, then forward, then
down.\actionnote{Extend the back}.

Keep arms in line with back (don't bring arms closer to legs than head
with neck straight).

Come up, change the crossing of the arms, and repeat.

\mbox{ }

\pose{\ardhal{}, to chair \PC{3}{11}}
\ardhalMarginFig{}

Focus on keeping back straight.

Move hands up back (towards shoulder blades).


\pose{\paschi{} \PC{3}{12}}
\paschiMarginFig{}


\pose{\newpose{\setubandsarv{}} (cross bolsters) \PC{3}{13}}
% \setubandsarvMarginFig{}

\setubandsarvFig{}
Get two bolsters - a round and a flat.

Put the round perpendicular across your mat

Put the flat bolster lengthwise over the round bolster, forming a
``+''

Lie down along top bolster so that head and shoulders come to
floor.

Stretch legs straight, heels resting on floor.

Chest lifted.\actionnote{Chest lifted}

Shoulders should be on floor, like in \ardhal{}.\linknote{Shoulders as
  in \ardhal{}}.

Arms out to the side.

\pose{\sav{}  \PC{3}{14}}
\savMarginFig{}

  
% \section{Sequence in Images for Week 3}
% \label{seqimags:3}

% \begin{tabular}{|c|c|c|c|}
% \tadFig{} &
% \urdbadFig{} &
% \vrkFig{} &
% \utttrikFig{} \\ \hline
% \viraiiFig{} &
% \uttparsvaFig{} &
% \viraiFig{} &
% \utkaFig{} \\ \hline
% \parsvoFig{} &
% \badhastuttFig{} &
% \ardhalFig{} &
% \setubandsarvFig{} \\ \hline
% \savFig{} & & &
% \end{tabular}

\week{4}
\label{week:4}

\textbf{Themes}: More flexibility (e.g. \utt{} full pose). More
towards \sarv{} (\ekapadsarv{} from \ardhal{} on chair).

\textbf{Reading}: 

\section{Sequence with Instructions for Week 4}
\label{seq:4}

\pose{\tad{} \PC{4}{1}}
\tadMarginFig{}

\pose{\urdbad{} \PC{4}{2}}
\urdbadMarginFig{}

\pose{\utttrik{} \PC{4}{3}}
\utttrikMarginFig{}

\pose{\viraii{} \PC{4}{4}}
\viraiiMarginFig{}

\pose{\uttparsva{} \PC{4}{5}}
\uttparsvaMarginFig{}

\newpage 
\pose{\newpose{\virai{}} \PC{4}{7}}
% \viraiMarginFig{}

\viraiFig{}
Start with \vim{}

Arms into \urdhast{}

Experiment with starting with hands on waist, then knee bend, then
\vim{} arms, versus starting with \utthastpad{} hands before
bending leg.

Turn front foot 90\textdegree out, back foot 60\textdegree.

Hands on waist

Bend front leg to a 90\textdegree angle

Keep trunk straight, perpendicular to floor.

Keep back heel down.

Straighten front leg, turn feet parallel,

Repeat on other side.


\pose{\utka{}  \PC{4}{8}}
\utkaMarginFig{}

\pose{\parsvo{} \PC{4}{9}}
\parsvoMarginFig{}

\pose{\newpose{\utt{}} (full pose) \PC{4}{10}}
% \uttMarginFig{}

\uttFig{}
Get two blocks in case needed

Feet apart

Start with \badhasttad{}

Then \badhastutt{}

Finally, extend arms to floor, using blocks if needed.

Come up. Feet together, \urdhast{} hands, extend, bend, bring
hands to floor or blocks.

Bend at hips, keeping back extended.

\pose{\ardhal{} (from chair) \PC{4}{11}}
\ardhalMarginFig{}


\pose{\newpose{\ekapadsarv{}} from \newline
  \ardhal{} on chair \PC{4}{12}}
% \ekapadsarvMarginFig{}

\ekapadsarvFig{}
From \ardhal{}, left right leg straight towards ceiling.

Keep back straight.

Bring right leg down, repeat with left leg.

\mbox{ }

\mbox{ }

\mbox{ }

\pose{\paschi{} \PC{4}{13}}
\paschiMarginFig{}

\pose{\setubandsarv{} \PC{4}{14}}
\setubandsarvMarginFig{}

\pose{\sav{} \PC{4}{15}}
\savMarginFig{}


% \section{Sequence in Images for Week 4}
% \label{seqimags:4}

% \begin{tabular}{|c|c|c|c|}
% \tadFig{} & 
% \urdbadFig{} & 
% \utttrikFig{} & 
% \viraiiFig{} \\ \hline
% \uttparsvaFig{} &
% \vimFig{} & 
% \viraiFig{} & 
% \utkaFig{} \\ \hline
% \parsvoFig{} & 
% \uttFig{} &
% \ardhalFig{} & 
% \ekapadsarvFig{} \\ \hline
% \paschiFig{} & 
% \setubandsarvFig{} & 
% \savFig{} &
% \end{tabular}


\week{5}
\label{week:5}

\textbf{Themes}: \ams{}! Arms behind back (e.g. \paschbadhast{},
\gomu{}, \paschnama{}). \sarv{} from chair.

\textbf{Reading}: 

\section{Sequence with Instructions for Week 5}
\label{seq:5}

\pose{\tad{} \PC{5}{1}}
\tadMarginFig{}

\pose{\urdhast{} \PC{5}{2}}
\urdhastMarginFig{}

\pose{\urdbad{}  \PC{5}{3}}
\urdbadMarginFig{}

\pose{\nam{}  \PC{5}{4}}
\namMarginFig{}

\pose{\urdnam{} from \newline
  \urdhast{}  \PC{5}{5}}
\urdnamMarginFig{}

\newpage

\pose{\newpose{\paschbadhast{}}  \PC{5}{6}}
% \paschbadhastMarginFig{}

\paschbadhastFig{}
Hands behind back, hold elbows.

Pull shoulder blades together.

Keep shoulders down, neck long, back straight.


\mbox{ }

\mbox{ }

\mbox{ }

\mbox{ }

\pose{\newpose{\gomu{}} (arms only) \PC{5}{7}}
% \gomuMarginFig{}

\gomuFig{}
Get a belt, drape over right shoulder.

Stand in \tad{}

Upper arm first. Right arm. Bend elbow, hand reaching down
back. Use left hand to gently push elbow back. Elbow pointing
straight up. Keep head up, facing forward.

Lower arm. Sweep left arm out and around to back. Bend elbow.

Can hands clasp? If not, grab belt with right hand, then left
hand. See if can walk hands towards each other along belt.

Head up, abdomen in.

\pose{\newpose{\paschnama{}}  \PC{5}{8}}
% \paschnamaMarginFig{}

\paschnamaFig{}
Move shoulder blades towards each other and into back, as in
\paschbadhast{}.\linknote{Shoulders and upper arms as in
  \paschbadhast{}} 

Scooch hands up back.

Can you rotate hands and arms so that thumbs come together?

\mbox{ }

\mbox{ }

\mbox{ }


\pose{\vrk{} \PC{5}{9}}
\vrkMarginFig{}

\newpage

\pose{\utka{} \PC{5}{10}}
\utkaMarginFig{}

\pose{\utthastpad{} \PC{5}{11}}
\utthastpadFig{}


\mbox{ }

\mbox{ }

\mbox{ }

\mbox{ }

\mbox{ }

\mbox{ }

\mbox{ }


\pose{\uttparsva{}  \PC{5}{12}}
\uttparsvaMarginFig{}

\pose{\utttrik{}  \PC{5}{13}}
\utttrikMarginFig{}

\pose{\viraii{} \PC{5}{14}}
\viraiiMarginFig{}

\pose{\uttparsva{}  \PC{5}{15}}

\uttparsvaFig{}

\mbox{ }

\mbox{ }

\mbox{ }

\mbox{ }

\mbox{ }

\mbox{ }


\mbox{ }

\pose{\vim{}  \PC{5}{16}}
\vimMarginFig{}

\pose{\virai{} \PC{5}{17}}
\viraiMarginFig{}

\newpage
\pose{\newpose{\praspad{}} (full pose)  \PC{5}{18}}
% \praspadMarginFig{}

\praspadFig{}
Have one or two blankets available, and a block.

Get into \praspadconc{}. 

Bring hands back in line with feet, fingers pointing forward. Back
still concave.

Keep elbows in as lower head to floor.\actionnote{Keep elbows in}

Elbows in, forearms perpendicular to floor. Shoulders away from ears,
neck long.

Add enough blankets and/or block to rest crown of head.

To come up, may need to scooch feet together a bit.

Transition to \praspadconc{}. Hands to waist. Come up.

\pose{\newpose{\ams{}}  \PC{5}{19}}
% \amsMarginFig{}

\amsFig{}
Mat to wall, stand against wall.

\utt{}, hands to floor. Walk hands forward 4 feet, keeping heels
at the wall.

Hands shoulder-width apart. Feet in line with hands.

Hands: weight balanced left hand vs right hand. For each hand,
weight balanced evenly front-to-back, pinky-to-thumb. Palms open,
fingers spread apart.

Arms: elbows straight

Lengthen spine, raise buttocks toward ceiling

legs straight, knees open

Come up on toes, left buttocks as high as possible.

Now keep buttocks up high while lengthening calves and ankle to
bring heels down towards floor.

\pose{\utt{} \PC{5}{20}}
\uttMarginFig{}

Feet hip width apart.

Use blocks if hands cannot rest comfortably on floor.

Knees straight

Weight even on feet front to back.

Repeat with feet together.

\newpage

\pose{\newpose{\padang{}}  \PC{5}{21}}
% \padangMarginFig{}

\padangFig{}
Use belt if cannot grab toes.

Start with concave back. Head looking up/forward.

Then release into head down.


\mbox{ }

\mbox{ }

\mbox{ }

\mbox{ }


\pose{\ardhal{} \PC{5}{22}}
\ardhalMarginFig{}

Teach use of belt, as preparation for \sarv{}.


\pose{\newpose{\ekapadsarv{}} (from \ardhal{})  \PC{5}{23}}
% \ekapadsarvMarginFig{}

\ekapadsarvFig{}
From \ardhal{}, lift right leg straight up.

Bring leg down; repeat with other leg.


\mbox{ }

\mbox{ }

\mbox{ }

\mbox{ }

\mbox{ }

\pose{\newpose{\sarv{}} (from \ardhal{})  \PC{5}{24}}
% \sarvMarginFig{}

\sarvFig{}
Staying in \ardhal{},

Bring right foot up into \ekapadsarv{}.

Bring left foot up to join right foot.

Straighten back using hands.

Bring hands higher on back (towards neck).


\mbox{ }

\pose{\newpose{\hal{}}  \PC{5}{25}}
% \halMarginFig{}

\halFig{}
Bring legs down to chair, into \ardhal{}.

Push chair away from head,

Bring legs down, bringing feet to floor.

Hips should be over head.


\mbox{ }

\pose{\paschi{}  \PC{5}{27}}
\paschiMarginFig{}

    (skipping \karn{})

\pose{\sav{}  \PC{5}{18}}
\savMarginFig{}


% \section{Sequence in Images for Week 5}
% \label{seqimags:5}

% \begin{tabular}{|c|c|c|c|}
% \tadFig{} & 
% \urdhastFig{} & 
% \urdbadFig{} & 
% \namFig{} \\ \hline
% \urdnamFig{} &
% \paschbadhastFig{} & 
% \gomuFig{} & 
% \paschnamaFig{} \\ \hline
% \vrkFig{} & 
% \utkaFig{} & 
% \utthastpadFig{}  & 
% \uttparsvaFig{} \\ \hline
% \utttrikFig{} & 
% \viraiiFig{} & 
% \uttparsvaFig{} & 
% \vimFig{} \\ \hline
% \viraiFig{} & 
% \praspadFig{} & 
% \amsFig{} & 
% \uttFig{} \\ \hline
% \padangFig{} & 
% \ardhalFig{} & 
% \ekapadsarvFig{} & 
% \sarvFig{} \\ \hline
% \halFig{} &
% \paschiFig{} & 
% \savFig{} &
% \end{tabular}

\week{6}
\label{week:6}

\textbf{Themes}: Full \sarv{}. Full \ardchand{}.

\textbf{Reading}: 

\section{Sequence with Instructions for Week 6}
\label{seq:6}

\pose{\tad{} \PC{6}{1}}
\tadMarginFig{}

\pose{\urdbad{}  \PC{6}{2}}
\urdbadMarginFig{}

\pose{\paschbadhast{}  \PC{6}{3}}
\paschbadhastMarginFig{}

\pose{\gomu{}  \PC{6}{4}}
\gomuMarginFig{}

\pose{\paschbadhast{}  \PC{6}{5}}
\paschbadhastMarginFig{}

\pose{\vrk{}  \PC{6}{6}}
\vrkMarginFig{}

\pose{\utka{}  \PC{6}{7}}
\utkaMarginFig{}

\pose{\utthastpad  \PC{6}{8}}

\pose{\parshastpad{}  \PC{6}{9}}
\parshastpadMarginFig{}

\pose{\utttrik{}  \PC{6}{10}}
\utttrikMarginFig{}

\pose{\viraii{}  \PC{6}{11}}
\viraiiMarginFig{}

\pose{\uttparsva{}  \PC{6}{12}}
\uttparsvaMarginFig{}

\pose{\vim{}  \PC{6}{13}}
\vimMarginFig{}

\pose{\virai{} \PC{6}{14}}
\viraiMarginFig{}

\newpage
\pose{\newpose{\ardchand{}}  \PC{6}{15}}
% \ardchandMarginFig{}

\ardchandFig{}
Have a block available.

\utttrik{}

Bend right knee, place right hand on floor about a foot in front
of right foot.

Left hand on waist

Bring left foot in a little towards right foot

Raise left leg so parallel to floor

Extend right leg

If cannot extend right leg with hand on floor, put right hand on
block.

Can you raise your left arm towards ceiling?

Repeat on other side.

If balance is an issue, can do this pose against wall.


\pose{\newpose{\parsvo{}} (full pose) \PC{6}{16}}
% \parsvoMarginFig{}

\parsvoFig{}

\mbox{ }

\mbox{ }

\mbox{ }

\mbox{ }

\mbox{ }

\mbox{ }


\pose{\ams{}  \PC{6}{17}}
\amsMarginFig{}

\pose{\utt{} \PC{6}{18,19}}
\uttMarginFig{}

Use blocks if needed

First with feet apart, then repeat with feet together.

\pose{\padang{}  \PC{6}{20}}
\padangMarginFig{}

\pose{\ardhal{}  \PC{6}{21}}
\ardhalMarginFig{}

\pose{\ekapadsarv{}  \PC{6}{22}}
\ekapadsarvMarginFig{}

\pose{\sarv{}  \PC{6}{23}}
\sarvMarginFig{}

\pose{\hal{}  \PC{6}{24}}
\halMarginFig{}

\pose{\paschi{}  \PC{6}{26}}
\paschiMarginFig{}

    skipping \karn{}

\pose{\sav{}  \PC{6}{27}}
\savMarginFig{}

% \newpage
% \section{Sequence in Images for Week 6}
% \label{seqimags:6}

% \begin{tabular}{|c|c|c|c|}
% \tadFig{} & 
% \urdbadFig{} & 
% \paschbadhastFig{} & 
% \gomuFig{} \\ \hline
% \paschbadhastFig{} &
% \vrkFig{} & 
% \utkaFig{} & 
% \utthastpadFig{} \\ \hline
% \parshastpadFig{} & 
% \utttrikFig{}  &
% \viraiiFig{} & 
% \uttparsvaFig{} \\ \hline
% \vimFig{} & 
% \viraiFig{} & 
% \ardchandFig{} &
% \parsvoFig{} \\ \hline
% \amsFig{} & 
% \uttFig{} & 
% \padangFig{} & 
% \ardhalFig{} \\ \hline
% \end{tabular}


% \begin{tabular}{|c|c|c|c|}
% \ekapadsarvFig{} & 
% \sarvFig{} & 
% \halFig{} & 
% \paschiFig{} \\ \hline
% \savFig{} & & & 
% \end{tabular}

\bibliography{yoga}
\bibliographystyle{alpha}


\end{document}
