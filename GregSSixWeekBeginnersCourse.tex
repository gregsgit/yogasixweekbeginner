\documentclass{book}

\usepackage{amsmath}
\usepackage{framed}
\usepackage{hyperref}
\usepackage[pdftex]{graphicx}
\usepackage{enumitem}
\usepackage{textcomp}   % for degree symbol
\usepackage{titlesec}
\usepackage{wrapfig}
\usepackage[dvipsnames]{xcolor}
\usepackage{xstring}


% Pose shortcodes
\newcommand{\apose}[1]{\emph{#1}}
%
\newcommand{\ams}{\apose{Adho Mukha Svanasana}}
\newcommand{\ardhal}{\apose{Ardha Halasana}}
\newcommand{\ardchand}{\apose{Ardha Chandrasana}}
\newcommand{\badd}{\apose{Baddhanguliyasana}}
\newcommand{\badhasttad}{\apose{Baddha Hasta Tadasana}}
\newcommand{\badhastutt}{\apose{Baddha Hasta Uttanasana}}
\newcommand{\dand}{\apose{Dandasana}}
\newcommand{\ekapadsarv}{\apose{Eka Pada Sarvangasana}}
\newcommand{\gomu}{\apose{Gomukasana}}
\newcommand{\hal}{\apose{Halasana}}
\newcommand{\karn}{\apose{Karnapidasana}}
\newcommand{\nam}{\apose{Namaskarasana}}
\newcommand{\padadand}{\apose{Padanghusta Dandasana}}
\newcommand{\padang}{\apose{Padangusthasana}}
\newcommand{\parshastpad}{Parsva Hasta Padasana}
\newcommand{\parsvo}{\apose{Parsvottanasana}}
\newcommand{\paschi}{\apose{Paschimottanasana}}
\newcommand{\paschbadhast}{\apose{Paschima Baddha Hastasana}}
\newcommand{\paschnama}{\apose{Paschima Namaskarasana}}
\newcommand{\praspad}{\apose{Prasarita Padottanasana}}
\newcommand{\sarv}{\apose{Salamba Sarvangasana}}
\newcommand{\sav}{\apose{Savasana}}
\newcommand{\setubandsarv}{\apose{Setu Bandha Sarvangasana}}
\newcommand{\tad}{\apose{Tadasana}}
\newcommand{\urdbad}{\apose{Urdhva Baddhanguliyasana}}
\newcommand{\urdhast}{\apose{Urdhva Hastasana}}
\newcommand{\urdhastdand}{\apose{Urdhva Hasta Dandasana}}
\newcommand{\urdnam}{\apose{Urdhva Namaskarasana}}
\newcommand{\utka}{\apose{Utkatasana}}
\newcommand{\utt}{\apose{Uttanasana}}
\newcommand{\utthastpad}{\apose{Utthita Hasta Padasana}}
\newcommand{\uttparsva}{\apose{Utthita Parsvakonasana}}
\newcommand{\utttrik}{\apose{Utthita Trikonasana}}
\newcommand{\vim}{\apose{Vimanasana}}
\newcommand{\virai}{\apose{Virabhdrasana I}}
\newcommand{\viraii}{\apose{Virabhadrasana II}}
\newcommand{\vrk}{\apose{Vrksasana}}

\newcommand{\poseFig}[1]{
  \begin{wrapfigure}{r}{2.1in}
    \begin{framed}
      \includegraphics[width=2.0in]{Figures/{"#1"}.jpg}
      \caption{{#1}}
    \end{framed}
  \end{wrapfigure}
}

\newcommand{\amsFig}{\poseFig{Adho Mukha Svanasana}}
\newcommand{\ardhalFig}{\poseFig{Ardha Halasana}}
\newcommand{\ardchandFig}{\poseFig{Ardha Chandrasana}}
\newcommand{\baddFig}{\poseFig{Baddhanguliyasana}}
\newcommand{\badhasttadFig}{\poseFig{Baddha Hasta Tadasana}}
\newcommand{\badhastuttFig}{\poseFig{Baddha Hasta Uttanasana}}
\newcommand{\dandFig}{\poseFig{Dandasana}}
\newcommand{\ekapadsarvFig}{\poseFig{Eka Pada Sarvangasana}}
\newcommand{\gomuFig}{\poseFig{Gomukasana}}
\newcommand{\halFig}{\poseFig{Halasana}}
\newcommand{\karnFig}{\poseFig{Karnapidasana}}
\newcommand{\namFig}{\poseFig{Namaskarasana}}
\newcommand{\padadandFig}{\poseFig{Padanghusta Dandasana}}
\newcommand{\padangFig}{\poseFig{Padangusthasana}}
\newcommand{\parshastpadFig}{Parsva Hasta Padasana}
\newcommand{\parsvoFig}{\poseFig{Parsvottanasana}}
\newcommand{\paschiFig}{\poseFig{Paschimottanasana}}
\newcommand{\paschbadhastFig}{\poseFig{Paschima Baddha Hastasana}}
\newcommand{\paschnamaFig}{\poseFig{Paschima Namaskarasana}}
\newcommand{\praspadFig}{\poseFig{Prasarita Padottanasana}}
\newcommand{\sarvFig}{\poseFig{Salamba Sarvangasana}}
\newcommand{\savFig}{\poseFig{Savasana}}
\newcommand{\setubandsarvFig}{\poseFig{Setu Bandha Sarvangasana}}
\newcommand{\tadFig}{\poseFig{Tadasana}}
\newcommand{\urdbadFig}{\poseFig{Urdhva Baddhanguliyasana}}
\newcommand{\urdhastFig}{\poseFig{Urdhva Hastasana}}
\newcommand{\urdhastdandFig}{\poseFig{Urdhva Hasta Dandasana}}
\newcommand{\urdnamFig}{\poseFig{Urdhva Namaskarasana}}
\newcommand{\utkaFig}{\poseFig{Utkatasana}}
\newcommand{\uttFig}{\poseFig{Uttanasana}}
\newcommand{\utthastpadFig}{\poseFig{Utthita Hasta Padasana}}
\newcommand{\uttparsvaFig}{\poseFig{Utthita Parsvakonasana}}
\newcommand{\utttrikFig}{\poseFig{Utthita Trikonasana}}
\newcommand{\vimFig}{\poseFig{Vimanasana}}
\newcommand{\viraiFig}{\poseFig{Virabhdrasana I}}
\newcommand{\viraiiFig}{\poseFig{Virabhadrasana II}}
\newcommand{\vrkFig}{\poseFig{Vrksasana}}

% \PC{week}{pose #}
\newcommand{\PC}[2]{{\normalfont\normalsize \hfill(Week #1, Pose #2 in PC)}}

\newcommand{\posenote}[1]{{\normalfont\normalsize \hfill(#1)}}

% \term{xxx} - introduce a standard Iyengar word or phrase
\newcommand{\term}[1]{``{#1}''}

% \newpose{xxx} - first time this pose has been used in course
% \newcommand{\newpose}[1]{{\color{Blue}{#1}}}
\newcommand{\newpose}[1]{{{#1}}}

\newcounter{week}
\newcounter{pose}

\newcommand{\week}[1]
{ \IfDecimal{#1}{\setcounter{week}{\integerpart}}{fooey}
  \setcounter{pose}{1}
  \chapter{Week {#1}}}

% a week is a chapter
\titleformat{\chapter}[frame]
{} % format
{} % label
{8pt} % sep
{\LARGE\bfseries\filcenter} % before-code
{} % after-code

% a pose is a subsection
\newcommand{\pose}{\subsection}
\titleformat{\subsection}[hang]
{\normalfont} % format
{\fbox{w\theweek.p\thepose}} % label
{0.5em} % sep
{\Large\bfseries\refstepcounter{pose}} % before-code
{} % after-code

\titlespacing{\subsection}
{-5.0ex} % left
{1.5ex plus .1ex minus .2ex} % before-sep
{1pc} % after-sep
% [] % right-sep


\setlength{\parindent}{0cm}
\setlength{\parskip}{1ex}


\title{Six Week Iyengar Yoga Course for Beginners}
\author{Greg Sullivan \href{mailto:gregs@sulliwood.org}{gregs@sulliwood.org} \\
For Heart of Iyengar Yoga Teacher Training course with Peentz Dubble
}
\date{March 11, 2017}

\begin{document}

\maketitle

\tableofcontents

\chapter*{Overview / Goals}
\label{chap:overview}

\addcontentsline{toc}{chapter}{Overview/Goals}

The overriding goal of a class for beginning students is to instill an
enjoyment and appreciation of yoga practice. The goal is to give
beginning students a taste for the benefits of yoga practice, and to,
as much as possible, give these beginning students tools that they can
use to safely and joyfully continue to practice yoga. As Geeta Iyengar
writes in the preface of ``Yoga in Action: A Preliminary
Course''\cite{GeetaPC}, 

\begin{quote}
  \emph{Yoga in Action for Beginners is not the end, but the beginning of
  yoga. It is for the practitioner to ignite the hidden force of yoga
  from within, so that it throws the Light on the path of the yogic
  journey.}
\end{quote}

Of course, Preliminary Course gives a 28-week syllabus, and for this
assignment we are creating only a 6-week course.

The immediate goals of the first few classes are to help students
familiarize themselves with their bodies, and also with the
terminology we use to describe motions and poses in Iyengar Yoga
classes.

Yoga, and in particular asana practice, is not about how you look. If
your pose is not geometrically perfect, you are not a failure. One of
the lessons that yoga teaches is vairagya, detachment. Along with
detachment is non-judgement. Judging yourself, in an asana, as ``good''
or ``bad'' is not helpful. The question is ``what is the goal?'' How do
you strive towards the goal? In this way, asana practice is a
microcosm of ``life practice''. The concept of ``bringing yoga off the
mat'' is important


I have mostly followed the sequences given for the first six weeks in
Geeta Iyengar's \emph{Preliminary Course}\cite{GeetaPC}, abbreviated
``PC''. For each pose given, I indicate whether it was in that week's
sequence in PC, and, if so, which number in the sequence in PC.
For example, if the fourth pose in week 2 is \nam{}, and it is also
the fourth pose in the week 2 syllabus in PC, I will write:

w2.4 \nam{} \PC{2}{4}

\section*{Formatting Notes}
\label{chap:formatting}

\addcontentsline{toc}{section}{Formatting Notes}

Pose names are italicized, as in \apose{this is a pose name}.

A new pose (not introduced earlier) will be both italicized and
colored blue, as in \newpose{\apose{this is a new pose}}.

Images of poses are taken from \emph{Preliminary Course}, as I do not
yet have images of myself in the poses.

\week{1}
\label{week:1}

\textbf{Themes}: What is yoga? Your body in space.

\textbf{Reading}: Something that relates asana to the rest of one's
life; how looking for alignment and balance in asana can be brought
``off the mat'' to bring alignment and balance in the rest of one's
life.

\section{Sequence for Week 1}
\label{seq:1}

\pose{\newpose{\sav{}} \posenote{not given as first pose in PC}}
\savFig

Have several blankets available if needed.

Lie flat; legs together, toes pointing towards ceiling. Knees pointing
to ceiling. Hips level. Feel what parts of your body are touching the
ground. Is there equal weight on both heels? Both buttocks? Shoulder
blades?


Now bring your arms over your head. Can you reach all the way to the
ground? Use blankets if need support. How close are hands; can you
bring them closer together? Rotate your arms so that your palms face
the ceiling. Now reverse the rotation; which causes the shoulder
blades to separate? Try rolling the upper arms (nearer the shoulders)
\term{inward} so that palms turn toward floor. Now keep upper arms
turned in, but make palms parallel.

Now try to increase the distance between top of shoulders and ears
(i.e. lower shoulders) while keeping arms straight above head.

\pose{\newpose{\tad{}}, against a wall \PC{1}{1}}
\tadFig

 Now we are going to attempt to replicate the pose we just did
 on the floor, with the help of gravity, while standing.

In \sav{}, the floor was a useful reference point -- we
could tell if our hips were even, or our shoulder blades were
even, by feeling the contact with the floor.

Stand with your back lightly touching the wall; your heels as
close to the wall as possible.

Have your feet together.

Is your weight balanced on your feet, left vs. right? Try
shifting all your weight to your left foot; now the right foot.

Is your weight evenly balanced forward and backward? That is, is the
weight on the balls of your feet the same as the weight on your heels?
Try moving forward, putting more weight on the balls of the feet. Now
try lifting the toes, putting all the weight on the heels. Now back to
even.

Consider your left foot. Is the weight evenly balanced on the ``4
corners'' of the foot? Same for right foot.

Remember what was touching the floor in Savasana? Lightly touching the
wall, see if you can replicate the touchpoints from Savasana. Your
heels, buttocks', shoulder blades, and back of head.


\pose{\newpose{\urdhast{}} \PC{1}{2}}
\urdhastFig{}

Now raise your arms above your head. This is called \urdhast{}
(arms). Recall where they were when you were on your back; can you
position them the same way without gravity? Turn your arms ``inward''
again, so that palms turn towards wall. Now keep upper arms turned,
but make palms parallel.

Lower shoulders away from ears.

\pose{\newpose{\urdbad{}} \PC{1}{3}}
\urdbadFig{}

Bring your arms down to pointing straight in front of you.

Clasp your hands. Note which thumb is on top.

Separate your wrists and rotate the thumb sides down. This is \badd{}.

Raise your arms, keeping your hands in \badd{}; Straighten your arms,
especially the elbows.

Lower your arms, change the interlock of your fingers so that other
thumb is on top, repeat.


\pose{\newpose{\nam{}} \PC{1}{4}}
\namFig{}

\pose{\newpose{\urdnam{}} from \urdhast{} \PC{1}{5}}
\urdnamFig{}


\textbf{Interlude}. So far everything has been perfectly straight and
balanced, left to right, top to bottom. Now we're going to branch out.

\pose{\newpose{Half \utt} (made up - not in PC)}


Bring mats to wall.

Stand legs-distance from wall. Lean over so that torso is parallel to
floor, forming right angle; hands touching wall. Above waist is Urdhva
Hastasana (rotated 90\textdegree). Below waist is \tad.


\pose{\newpose{\utthastpad} \PC{1}{6}}
\utthastpadFig{}

Left foot to wall.

Separate feet 4-5 feet apart. Arms out, parallel to floor. Even
balance front-to-back, left to right.

Lift trunk and chest. Lower shoulders

\pose{\newpose{\parshastpad} \PC{1}{7}}
\parshastpad{}


Turn right leg out 90\textdegree. Right heel in line with center of
left foot (both feet centered on mat). Toes pointing 90\textdegree,
left knee pointing 90\textdegree. Turn left foot in
slightly. Everything else unchanged. Hips still parallel to long edge
of mat; torso facing forward; face facing forward. Are hips even (same
height)?

\pose{\newpose{\utttrik} \PC{1}{8}}
\utttrikFig{}


Extend the right arm out as you bend at the waist and bring the right
arm down to the shin. Place your left arm on your left waist. Keep
your torso perpendicular to the ground, requiring rotation
right-to-left of the torso. Now bring left arm up, pointing to
ceiling; two arms going in opposite directions. Lower shoulders
(increase distance between ears and shoulders). Are hips perpendicular
to the wall?

Come up, turn feet parallel.

Optional variation: Can try parallel to wall, so that wall gives
feedback to buttocks, shoulder blades, and head about alignment.

(repeat Utthita Hasta Padasana, Parsva Hasta Padasana, Utthita
Trikonasana) on opposite side (left foot to the wall).

\pose{\newpose{\parsvo{} (concave back) \PC{1}{9}}}
\parsvoFig

Grab two blocks.

Left foot to wall, Parsva Hasta Padasana. Now turn back foot in more,
and rotate hips to parallel wall. Hands to hips. Balance pelvis -
front-to-back and side-to-side. Chest facing front of mat, head facing
straight ahead.

Extend back, make concave, look up.

Now bend so that torso is parallel to floor (recall half-uttanasana).

Put blocks at whatever height allows you to keep concave back.

Repeat with right foot to wall.

\pose{\newpose{\praspad} (concave back) \PC{1}{10}}
\praspadFig

Wide separation of feet. Feet parallel to each other; even
w.r.t. distance from edge of mat.

Bend at waist so that torso is parallel to floor. Back
concave. Hands directly below shoulders on floor. Use blocks if
needed to get torso parallel to floor.

Hips over (in line with) feet. Weight even front-to-back on
feet. Feet flat, weight even inner-to-outer on feet.

\pose{\newpose{\dand}  \PC{1}{11}}
\dandFig

\pose{\newpose{\urdhastdand} \PC{1}{12}}
\urdhastdandFig

\pose{\newpose{\padadand} \PC{1}{13}}
\padadandFig

Use a belt for integrity of back.

Concave back

\pose{\newpose{\paschi} \PC{1}{14}}
\paschiFig

This is a good time to remind students to not be attached to the
perfect form/geometry of the asana, but to consider the goals.

\pose{\sav}
\savFig


\week{2}
\label{week:2}

\textbf{Themes}: 

\textbf{Reading}:


\section{Sequence for Week 2}
\label{seq:2}

\pose{\tad{} \PC{2}{1}}
\tadFig

\pose{\urdhast{} \PC{2}{2}}
\urdhastFig

\pose{\urdbad{} \PC{2}{3}}
\urdbadFig

\pose{\nam{} \PC{2}{4}}
\namFig{}

\pose{\urdnam{} \PC{2}{5}}
\urdnamFig{}


\pose{\utthastpad{} \PC{2}{6}}
\utthastpadFig{}

\pose{\parshastpad{} \PC{2}{7}}
\parshastpadFig{}

\pose{\utttrik{} \PC{2}{8}}
\utttrikFig{}

\pose{\newpose{\viraii{}} \PC{2}{9}}
\viraiiFig{}

Start in \parshastpad{} - as in start of \utttrik{}.

(do pose with hands on waist first, then with arms extended)

Keeping everything stable, bend right knee to 90\textdegree

Extend inner thigh; pull back outer thigh.

Look over front arm.

Repeat on other side

\pose{\newpose{\uttparsva{}} \PC{2}{10}}
\uttparsvaFig{}


Again, start in \parshastpad{}

Again, bend right knee to 90\textdegree

Left hand on waist.

Bring right hand to floor, or block, keeping chest facing wall in
front of you (perpendicular to floor).

Rotate abdomen.

Raise left arm towards ceiling.

Rotate left arm as in \utthastpad{}

See if you can bring arm down alongside ear.

\pose{\parsvo{} (standing, then concave back) \PC{2}{11}}
\parsvoFig{}

\pose{\newpose{\parsvo{}} \PC{2}{12}}
\parsvoFig{}

Bend at waist, sternum directly over front knee, put hands on
blocks (at highest setting), keeping back concave.

Now walk blocks forward, extending back.

Finally, bring head down over knee.

\pose{\praspad{} \PC{2}{13}}
\praspadFig{}

\pose{\dand{} \PC{2}{14}}
\dandFig{}

\pose{\urdhastdand{} \PC{2}{15}}
\urdhastdandFig{}

\pose{\padadand{} \PC{2}{16}}
\padadandFig{}

\pose{\newpose{\ardhal{}} to chair (P.51) \PC{2}{17}}
\ardhalFig{}

Get 4 blankets, a block, and a chair

3 blankets - set up as for Sarvangasana.

Chair at head, 1 blanket on chair. block for tailbone.

Shoulders 2 in.s from edge of blankets.

Swing up.

Goals: hips over head, legs parallel to floor.

Arm positions for Sarvangasna (w/o belts). Elbows down and
in. Hands on mid-to-upper back. Use hands to lift hips and
straighten back.

\pose{\paschi{} \PC{2}{18}}
\paschiFig{}

\pose{\sav{} \PC{2}{19}}
\savFig{}
  
\week{3}
\label{week:3}

\textbf{Themes}: 

\textbf{Reading}: 

\section{Sequence for Week 3}
\label{seq:3}

\pose{\tad{} \PC{3}{1}}
\tadFig{}

\pose{\urdbad{} \PC{3}{2}}
\urdbadFig{}

\pose{\newpose{\vrk{}} \PC{3}{3}}
\vrkFig{}


Stand with back to wall, far enough away so that can reach back to
touch wall for balance.

Bend the right knee, keeping left leg in Tadasana, grab right foot
with right hand.

Place right foot high on inside of left thigh.

Push thigh out against foot, foot in against thigh.

Push knee back - lengthen inner right thigh, pull in outer right
thigh.

Can you get your knee parallel with wall (perpendicular to gaze)?

Balance free of wall

Can you raise your left hand straight up?

Now your right hand?

Can you bring your hands together into Urdhva Namaskarasana?

Lower arms, lower right leg, repeat with left leg.

\pose{\utttrik{} \PC{3}{4}}
\utttrikFig{}

\pose{\viraii{} \PC{3}{5}}
\viraiiFig{}

\pose{\uttparsva{} \PC{3}{6}}
\uttparsvaFig{}

\pose{\newpose{\virai{}} (turning the trunk) \PC{3}{7}}
\viraiFig{}

Stand left foot to wall. Turn front foot 90\textdegree out, back
foot in 60\textdegree.

Turn the trunk to face away from wall.

Focus on turning pelvis to face evenly perpendicular to mat.

With left hand, grab left thigh and pull/rotate it forward, to
help turn the pelvis.

Repeat on other side.

\pose{\newpose{\utka{}} (arms first) \PC{3}{8}}
\utkaFig{}


Arms up straight in \urdnam{}.

Try to keep elbows straight and bring palms as close together as
possible.

Bend the knees until thighs are parallel to floor. Keep heels
down.

\pose{\parsvo{} \PC{3}{9}}
\parsvoFig{}

Start with concave stage (use blocks if necessary)

Then head down

\pose{\newpose{\badhastutt{}} \PC{3}{10}}
\badhastuttFig{}


Start in \tad{}, feet apart.

\badhasttad{}

Exhale, stretch the trunk up, then forward, then down.

Come up, change the crossing of the arms, and repeat.

\pose{\ardhal{}, to chair \PC{3}{11}}
\ardhalFig{}

\pose{\paschi{} \PC{3}{12}}

\pose{\newpose{setubandsarv{}} (cross bolsters) \PC{3}{13}}
setubandsarvFig{}


Get two bolsters - a round and a flat.

Put the round perpendicular across your mat

Put the flat bolster lengthwise over the round bolster, forming a
``+''

Lie down along top bolster so that head and shoulders come to
floor.

Stretch legs straight, heels resting on floor.

Arms out to the side.

\pose{\sav{}  \PC{3}{14}}
\savFig{}


\week{4}
\label{week:4}

\textbf{Themes}: 

\textbf{Reading}: 

\section{Sequence for Week 4}
\label{seq:4}

\pose{\tad{} \PC{4}{1}}
\tadFig{}

\pose{\urdbad{} \PC{4}{2}}
\urdbadFig{}

\pose{\utttrik{} \PC{4}{3}}
\utttrikFig{}

\pose{\viraii{} \PC{4}{4}}
\viraiiFig{}

\pose{\uttparsva{} \PC{4}{5}}
\uttparsvaFig{}

\pose{\newpose{\vim{}} \PC{4}{6}}
\vimFig{}


Left foot to wall

Turn front foot 90\textdegree out, back foot 60\textdegree.

Hands on waist

Bend front leg to a 90\textdegree angle

Keep trunk straight, perpendicular to floor.

Keep back heel down.

Straighten front leg, turn feet parallel,

Repeat on other side.

\pose{\newpose{\virai{}} \PC{4}{7}}
\viraiFig{}


Start with \vim{}

Arms into \urdhast{}

Experiment with starting with hands on waist, then knee bend, then
\vim{} arms, versus starting with \utthastpad{} hands before
bending leg.

\pose{\utka{}  \PC{4}{8}}
\utkaFig{}

\pose{\parsvo{} \PC{4}{9}}
\parsvoFig{}

\pose{\newpose{\utt{}} (full pose) \PC{4}{10}}
\uttFig{}

Get two blocks in case needed

Feet apart

Start with \badhasttad{}

Then \badhastutt{}

Finally, extend arms to floor, using blocks if needed.

Come up. Feet together, \urdhast{} hands, extend, bend, bring
hands to floor or blocks.

Bend at hips, keeping back extended.

\pose{\ardhal{} (from chair) \PC{4}{11}}
\ardhalFig{}

\pose{\newpose{\ekapadsarv{}} (from \ardhal{} on chair) \PC{4}{12}}
\ekapadsarvFig{}

From \ardhal{}, left right leg straight towards ceiling.

Keep back straight.

Bring right leg down, repeat with left leg.

\pose{\paschi{} \PC{4}{13}}
\paschiFig{}

\pose{\setubandsarv{} \PC{4}{14}}
\setubandsarvFig{}

\pose{\sav{} \PC{4}{15}}
\savFig{}


\week{5}
\label{week:5}

\textbf{Themes}: Consolidation, \ams{}!, \sarv{}!

\textbf{Reading}: 

\section{Sequence for Week 5}
\label{seq:5}

\pose{\tad{} \PC{5}{1}}
\tadFig{}

\pose{\urdhast{} \PC{5}{2}}
\urdhastFig{}

\pose{\urdbad{}  \PC{5}{3}}
\urdbadFig{}

\pose{\nam{}  \PC{5}{4}}
\namFig{}

\pose{\urdnam{} from \urdhast{}  \PC{5}{5}}
\urdnamFig{}

\pose{\newpose{\paschbadhast{}}  \PC{5}{6}}
\paschbadhastFig{}

Hands behind back, hold elbows.

\pose{\newpose{\gomu{}} (arms only) \PC{5}{7}}
\gomuFig{}


Get a belt, drape over right shoulder.

Stand in \tad{}

Upper arm first. Right arm. Bend elbow, hand reaching down
back. Use left hand to gently push elbow back. Elbow pointing
straight up. Keep head up, facing forward.

Lower arm. Sweep left arm out and around to back. Bend elbow.

Can hands clasp? If not, grab belt with right hand, then left
hand. See if can walk hands towards each other along belt.

Head up, abdomen in.

\pose{\newpose{\paschnama{}}  \PC{5}{8}}
\paschnamaFig{}


Move shoulder blades towards each other and into back.

Scooch hands up back.

Can you rotate hands and arms so that thumbs come together?

\pose{\vrk{} \PC{5}{9}}
\vrkFig{}

\pose{\utka{} \PC{5}{10}}
\utkaFig{}

\pose{\utthastpad{} \PC{5}{11}}
\utthastpadFig{}

\pose{\uttparsva{}  \PC{5}{12}}
\uttparsvaFig{}

\pose{\utttrik{}  \PC{5}{13}}
\utttrikFig{}

\pose{\viraii{} \PC{5}{14}}
\viraiiFig{}

\pose{\uttparsva{}  \PC{5}{15}}
\uttparsvaFig{}

\pose{\vim{}  \PC{5}{16}}
\vimFig{}

\pose{\virai{} \PC{5}{17}}
\viraiFig{}

\pose{\newpose{\praspad{}} (full pose)  \PC{5}{18}}
\praspadFig{}

\pose{\newpose{\ams{}}  \PC{5}{19}}
\amsFig{}


Mat to wall, stand against wall.

\utt{}, hands to floor. Walk hands forward 4 feet, keeping heels
at the wall.

Hands shoulder-width apart. Feet in line with hands.

Hands: weight balanced left hand vs right hand. For each hand,
weight balanced evenly front-to-back, pinky-to-thumb. Palms open,
fingers spread apart.

Arms: elbows straight

Lengthen spine, raise buttocks toward ceiling

legs straight, knees open

Come up on toes, left buttocks as high as possible.

Now keep buttocks up high while lengthening calves and ankle to
bring heels down towards floor.

\pose{\newpose{\utt{}} (concave back)  \PC{5}{20}}
\uttFig{}


Get 2 blocks

Feet hip width apart.

use blocks if hands cannot rest comfortably on floor.

Knees straight

Weight even on feet front to back.

repeat with feet together.

\pose{\newpose{\padang{}}  \PC{5}{21}}
\padangFig{}


Use belt if cannot grab toes.

Start with concave back. Head looking up/forward.

Then release into head down.

\pose{\ardhal{} (with chair)  \PC{5}{22}}
\ardhalFig{}


Teach use of belt, as preparation for \sarv{}.

\pose{\newpose{\ekapadsarv{}} (from \ardhal{})  \PC{5}{23}}
\ekapadsarvFig{}


\pose{\newpose{\sarv{}} (from \ardhal{})  \PC{5}{24}}
\sarvFig{}


Staying in \ardhal{},

Bring right foot up into \ekapadsarv{}.

Bring left foot up to join right foot.

Straighten back using hands.

Bring hands higher on back (towards neck).

\pose{\newpose{\hal{}}  \PC{5}{25}}
\halFig{}


Bring legs down to chair, into \ardhal{}.

Push chair away from head,

Bring legs down, bringing feet to floor.

Hips should be over head.

\pose{\paschi{}  \PC{5}{27}}
\paschiFig{}

    (skipping \karn{})

\pose{\sav{}  \PC{5}{18}}
\savFig{}


\week{6}
\label{week:6}

\textbf{Themes}: \ardchand{}!

\textbf{Reading}: 

\section{Sequence for Week 6}
\label{seq:6}

\pose{\tad{} \PC{6}{1}}
\tadFig{}

\pose{\urdbad{}  \PC{6}{2}}
\urdbadFig{}

\pose{\paschbadhast{}  \PC{6}{3}}
\paschbadhastFig{}

\pose{\gomu{}  \PC{6}{4}}
\gomuFig{}

\pose{\paschbadhast{}  \PC{6}{5}}
\paschbadhastFig{}

\pose{\vrk{}  \PC{6}{6}}
\vrkFig{}

\pose{\utka{}  \PC{6}{7}}
\utkaFig{}

\pose{\utthastpad  \PC{6}{8}}
\utthastpadFig{}

\pose{\parshastpad{}  \PC{6}{9}}
\parshastpadFig{}

\pose{\utttrik{}  \PC{6}{10}}
\utttrikFig{}

\pose{\viraii  \PC{6}{11}}
\viraiiFig{}

\pose{\uttparsva  \PC{6}{12}}
\uttparsvaFig{}

\pose{\vim  \PC{6}{13}}
\vimFig{}

\pose{\virai \PC{6}{14}}
\viraiFig{}

\pose{\newpose{\ardchand{}}  \PC{6}{15}}
\ardchandFig{}


Have a block available.

\utttrik{}

Bend right knee, place right hand on floor about a foot in front
of right foot.

Left hand on waist

Bring left foot in a little towards right foot

Raise left leg so parallel to floor

Extend right leg

If cannot extend right leg with hand on floor, put right hand on
block.

Can you raise your left arm towards ceiling?

Repeat on other side.

\pose{\newpose{\parsvo{}} (full pose) \PC{6}{16}}
\parsvoFig{}

\pose{\ams{}  \PC{6}{17}}
\amsFig{}

\pose{\utt{} (concave back)  \PC{6}{18,19}}
\uttFig{}


Use blocks if neede

First with feet apart, then repeat with feet together.

\pose{\padang{}  \PC{6}{20}}
\padangFig{}

\pose{\ardhal{}  \PC{6}{21}}
\ardhalFig{}

\pose{\ekapadsarv{}  \PC{6}{22}}
\ekapadsarvFig{}

\pose{\sarv{}  \PC{6}{23}}
\sarvFig{}

\pose{\hal{}  \PC{6}{24}}
\halFig{}

\pose{\paschi{}  \PC{6}{26}}
\paschiFig{}

    skipping \karn{}

\pose{\sav  \PC{6}{27}}
\savFig{}


\bibliography{yoga}
\bibliographystyle{alpha}


\end{document}
